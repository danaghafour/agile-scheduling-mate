
%%
%% forked from https://gits-15.sys.kth.se/giampi/kthlatex kthlatex-0.2rc4 on 2020-02-13
%% expanded upon by Gerald Q. Maguire Jr.
%% This template has been adapted by Anders Sjögren to the University
%% Engineering Program in Computer Science at KTH ICT. This adaptation was
%% translation of English headings into Swedish with the addition of Swedish.
%% Many thanks to others who have provided constructive input regarding the template.

% Make it possible to conditionally depend on the TeX engine used
\RequirePackage{ifxetex}
\RequirePackage{ifluatex}
\newif\ifxeorlua
\ifxetex\xeorluatrue\fi
\ifluatex\xeorluatrue\fi

\ifxeorlua
% The following is to ensure that the PDF uses a recent version rather than the typical PDF 1-5
%  This same version of PDF should be set as an option for hyperef

\RequirePackage{expl3}
\ExplSyntaxOn
%pdf_version_gset:n{2.0}
%\pdf_version_gset:n{1.5}

%% Alternatively, if you have a LaTeX newer than June 2022, you can use the following. However, then you have to remove the pdfversion from hyperef. It also breaks hyperxmp. So perhaps it is too early to try using it!
%\DocumentMetadata
%{
%% testphase = phase-I, % tagging without paragraph tagging
% testphase = phase-II % tagging with paragraph tagging and other new stuff.
%pdfversion = 2.0 % pdfversion must be set here.
%}

% Optionally, you can set the uncompress flag to make it easier to examine the PDF
%\pdf_uncompress: % to check the pdf
\ExplSyntaxOff
\else
\RequirePackage{expl3}
\ExplSyntaxOn
%\pdf_version_gset:n{2.0}
% \pdf\_version\_gset:n{1.5}
\pdfminorversion=5
\ExplSyntaxOff
\fi

%% Define a pair of commands to disable and reenable specific packages - see https://tex.stackexchange.com/questions/39415/unload-a-latex-package
\makeatletter
\newcommand{\disablepackage}[2]{%
  \disable@package@load{#1}{#2}%
}
\newcommand{\reenablepackage}[1]{%
  \reenable@package@load{#1}%
}
\makeatother
%% To avoid the warning: "Package transparent Warning: Loading aborted, because pdfTeX is not running in PDF mode."
\ifxeorlua
\disablepackage{transparent}{}
\fi

%% The template is designed to handle a thesis in English or Swedish
% set the default language to english or swedish by passing an option to the documentclass - this handles the inside title page
% To optimize for digital output (this changes the color palette add the option: digitaloutput
% To use \ifnomenclature add the option nomenclature
% To use bibtex or biblatex - include one of these as an option
\documentclass[nomenclature, english, bibtex]{kththesis}
%\documentclass[swedish, biblatex]{kththesis}
% if pdflatex \usepackage[utf8]{inputenc}

%% Conventions for todo notes:
% Informational
%% \generalExpl{Comments/directions/... in English}
\newcommand*{\generalExpl}[1]{\todo[inline]{#1}}                

% Language-specific information (currently in English or Swedish)
\newcommand*{\engExpl}[1]{\todo[inline, backgroundcolor=kth-lightgreen40]{#1}} %% \engExpl{English descriptions about formatting}
\newcommand*{\sweExpl}[1]{\todo[inline, backgroundcolor=kth-lightblue40]{#1}}  %% % \sweExpl{Text på svenska}

% warnings
\newcommand*{\warningExpl}[1]{\todo[inline, backgroundcolor=kth-lightred40]{#1}} %% \warningExpl{warnings}

% Uncomment to hide specific comments, to hide **all** ToDos add `final` to
% document class
% \renewcommand\warningExpl[1]{}
% \renewcommand\generalExpl[1]{}
% \renewcommand\engExpl[1]{}
% For example uncommenting the following line hides the Swedish language explanations
% \renewcommand\sweExpl[1]{}


% \usepackage[style=numeric,sorting=none,backend=biber]{biblatex}
\ifbiblatex
    %\usepackage[language=english,bibstyle=authoryear,citestyle=authoryear, maxbibnames=99]{biblatex}
    % Alternatively you might use another style, such as IEEE and use citestyle=numeric-comp  to put multiple citations in a single pair of square brackets
    \usepackage[style=ieee,citestyle=numeric-comp]{biblatex}
    \addbibresource{references.bib}
    %\DeclareLanguageMapping{norsk}{norwegian}
\else
    % The line(s) below are for BibTeX
    \bibliographystyle{bibstyle/myIEEEtran}
    %\bibliographystyle{apalike}
\fi


% include a variety of packages that are useful
\input{lib/includes}
\input{lib/kthcolors}

%\glsdisablehyper
%\makeglossaries
%\makenoidxglossaries
%\input{lib/acronyms}                %load the acronyms file

\input{lib/defines}  % load some additional definitions to make writing more consistent

% The following is needed in conjunction with generating the DiVA data with abstracts and keywords using the scontents package and a modified listings environment
%\usepackage{listings}   %  already included
\ExplSyntaxOn
\newcommand\typestoredx[2]{\expandafter\__scontents_typestored_internal:nn\expandafter{#1} {#2}}
\ExplSyntaxOff
\makeatletter
\let\verbatimsc\@undefined
\let\endverbatimsc\@undefined
\lst@AddToHook{Init}{\hyphenpenalty=50\relax}
\makeatother


\lstnewenvironment{verbatimsc}
    {
    \lstset{%
        basicstyle=\ttfamily\tiny,
        backgroundcolor=\color{white},
        %basicstyle=\tiny,
        %columns=fullflexible,
        columns=[l]fixed,
        language=[LaTeX]TeX,
        %numbers=left,
        %numberstyle=\tiny\color{gray},
        keywordstyle=\color{red},
        breaklines=true,                 % sets automatic line breaking
        breakatwhitespace=true,          % sets if automatic breaks should only happen at whitespace
        %keepspaces=false,
        breakindent=0em,
        %fancyvrb=true,
        frame=none,                     % turn off any box
        postbreak={}                    % turn off any hook arrow for continuation lines
    }
}{}

%% Add some more keywords to bring out the structure more
\lstdefinestyle{[LaTeX]TeX}{
morekeywords={begin, todo, textbf, textit, texttt}
}

%% definition of new command for bytefield package
\newcommand{\colorbitbox}[3]{%
	\rlap{\bitbox{#2}{\color{#1}\rule{\width}{\height}}}%
	\bitbox{#2}{#3}}




% define a left aligned table cell that is ragged right
\newcolumntype{L}[1]{>{\raggedright\let\newline\\\arraybackslash\hspace{0pt}}p{#1}}

% Because backref is not compatible with biblatex
\ifbiblatex
    \usepackage[plainpages=false]{hyperref}
\else
    \usepackage[
    backref=page,
    pagebackref=false,
    plainpages=false,
                            % PDF related options
    unicode=true,           % Unicode encoded PDF strings
    bookmarks=true,         % generate bookmarks in PDF files
    bookmarksopen=false,    % Do not automatically open the bookmarks in the PDF reading program
    pdfpagemode=UseNone,    % None, UseOutlines, UseThumbs, or FullScreen
    destlabel,              % better naming of destinations
    pdfencoding=auto,       % for unicode in 
    ]{hyperref}
    \makeatletter
    \ltx@ifpackageloaded{attachfile2}{
    % cannot use backref if one is using attachfile
    }
    {\usepackage{backref}
    %
    % Customize list of backreferences.
    % From https://tex.stackexchange.com/a/183735/1340
    \renewcommand*{\backref}[1]{}
    \renewcommand*{\backrefalt}[4]{%
    \ifcase #1%
          \or [Page~#2.]%
          \else [Pages~#2.]%
    \fi%
    }
    }
    \makeatother

\fi
\usepackage[all]{hypcap}	%% prevents an issue related to hyperref and caption linking

%% Acronyms
% note that nonumberlist - removes the cross references to the pages where the acronym appears
% note that super will set the descriptions text aligned
% note that nomain - does not produce a main glossary, thus only acronyms will be in the glossary
% note that nopostdot - will prevent there being a period at the end of each entry
\usepackage[acronym, style=super, section=section, nonumberlist, nomain,
nopostdot]{glossaries}
\setlength{\glsdescwidth}{0.75\textwidth}
\usepackage[]{glossaries-extra}
\ifinswedish
    %\usepackage{glossaries-swedish}
\fi

%% For use with the README_notes
% Define a new type of glossary so that the acronyms defined in the README_notes document can be distinct from those in the thesis template
% the tlg, tld, and dn will be the file extensions used for this glossary
\newglossary[tlg]{readme}{tld}{tdn}{README acronyms}


\input{lib/includes-after-hyperref}

%\glsdisablehyper
\makeglossaries
%\makenoidxglossaries

% The following bit of ugliness is because of the problems PDFLaTeX has handling a non-breaking hyphen
% unless it is converted to UTF-8 encoding.
% If you do not use such characters in your acronyms, this could be simplified to just include the acronyms file.
\ifxeorlua
\input{lib/acronyms}                %load the acronyms file
\else
\input{lib/acronyms-for-pdflatex}
\fi


% insert the configuration information with author(s), examiner, supervisor(s), ...
\input{custom_configuration}

\title{Smart Scheduling System for Optimized Workforce
Management}

% give the alternative title - i.e., if the thesis is in English, then give a Swedish title
\alttitle{Smart schemaläggningssystem för optimal personalhantering}
% alternative, if the thesis is in Swedish, then give an English title
%\alttitle{This is the English translation of the title}
%\altsubtitle{This is the English translation of the subtitle}

% Enter the English and Swedish keywords here for use in the PDF metadata _and_ for later use
% following the respective abstract.
% Try to put the words in the same order in both languages to facilitate matching. For example:
\EnglishKeywords{Employee Scheduling, Workforce Optimization, Constraint Solver, Timefold, Shift Management, Docker Deployment, Supabase, AI-Assisted UI Design, Human-Centered Design, Lovable}
\SwedishKeywords{Personalschemaläggning, arbetskraftsoptimering, begränsningslösare, Timefold, skifthantering, Docker-distribution, Supabase, AI-assisterad UI-design, människocentrerad design, Lovable}

%%%%% For the oral presentation
%% Add this information once your examiner has scheduled your oral presentation
\presentationDateAndTimeISO{2022-03-15 13:00}
\presentationLanguage{eng}
\presentationRoom{via Zoom https://kth-se.zoom.us/j/ddddddddddd}
\presentationAddress{Isafjordsgatan 22 (Kistagången 16)}
\presentationCity{Stockholm}

% When there are multiple opponents, separate their names with '\&'
% Opponent's information
\opponentsNames{A. B. Normal \& A. X. E. Normalè}

% Once a thesis is approved by the examiner, add the TRITA number
% The TRITA number for a thesis consists of two parts: a series (unique to each school)
% and the number in the series, which is formatted as the year followed by a colon and
% then a unique series number for the thesis - starting with 1 each year.
\trita{TRITA -- EECS-EX}{2024:0000}

% Put the title, author, and keyword information into the PDF meta information
\input{lib/pdf_related_includes}


% the custom colors and the commands are defined in defines.tex    
\hypersetup{
	colorlinks  = true,
	breaklinks  = true,
	linkcolor   = \linkscolor,
	urlcolor    = \urlscolor,
	citecolor   = \refscolor,
	anchorcolor = black
}

\ifnomenclature
% The following lines make the page numbers and equations hyperlinks in the Nomenclature list
\renewcommand*{\pagedeclaration}[1]{\unskip, \dotfill\hyperlink{page.#1}{page\nobreakspace#1}}
% The following does not work correctly, as the name of the cross-reference is incorrect
%\renewcommand*{\eqdeclaration}[1]{, see equation\nobreakspace(\hyperlink{equation.#1}{#1})}

% You can also change the page heading for the nomenclature
\renewcommand{\nomname}{List of Symbols Used}

% You can even add customization text before the list
\renewcommand{\nompreamble}{The following symbols will be later used within the body of the thesis.}
\makenomenclature
\fi

%
% The commands below are to configure JSON listings
% 
% format for JSON listings
\colorlet{punct}{red!60!black}
\definecolor{delim}{RGB}{20,105,176}
\definecolor{numb}{RGB}{106, 109, 32}
\definecolor{string}{RGB}{0, 0, 0}

\lstdefinelanguage{json}{
    numbers=none,
    numberstyle=\small,
    frame=none,
    rulecolor=\color{black},
    showspaces=false,
    showtabs=false,
    breaklines=true,
    postbreak=\raisebox{0ex}[0ex][0ex]{\ensuremath{\color{gray}\hookrightarrow\space}},
    breakatwhitespace=true,
    basicstyle=\ttfamily\small,
    extendedchars=false,
    upquote=true,
    morestring=[b]",
    stringstyle=\color{string},
    literate=
     *{0}{{{\color{numb}0}}}{1}
      {1}{{{\color{numb}1}}}{1}
      {2}{{{\color{numb}2}}}{1}
      {3}{{{\color{numb}3}}}{1}
      {4}{{{\color{numb}4}}}{1}
      {5}{{{\color{numb}5}}}{1}
      {6}{{{\color{numb}6}}}{1}
      {7}{{{\color{numb}7}}}{1}
      {8}{{{\color{numb}8}}}{1}
      {9}{{{\color{numb}9}}}{1}
      {:}{{{\color{punct}{:}}}}{1}
      {,}{{{\color{punct}{,}}}}{1}
      {\{}{{{\color{delim}{\{}}}}{1}
      {\}}{{{\color{delim}{\}}}}}{1}
      {[}{{{\color{delim}{[}}}}{1}
      {]}{{{\color{delim}{]}}}}{1}
      {’}{{\char13}}1,
}

\lstdefinelanguage{XML}
{
  basicstyle=\ttfamily\color{blue}\bfseries\small,
  morestring=[b]",
  morestring=[s]{>}{<},
  morecomment=[s]{<?}{?>},
  stringstyle=\color{black},
  identifierstyle=\color{blue},
  keywordstyle=\color{cyan},
  breaklines=true,
  postbreak=\raisebox{0ex}[0ex][0ex]{\ensuremath{\color{gray}\hookrightarrow\space}},
  breakatwhitespace=true,
  morekeywords={xmlns,version,type}% list your attributes here
}

% In case you use both listings and lstlistings - this makes them both use the same counter
\makeatletter
\AtBeginDocument{\let\c@listing\c@lstlisting}
\makeatother
\usepackage{subfiles}

% To have Creative Commons (CC) license and logos use the doclicense package
% Note that the lowercase version of the license has to be used in the modifier
% i.e., one of by, by-nc, by-nd, by-nc-nd, by-sa, by-nc-sa, zero.
% For background see:
% https://www.kb.se/samverkan-och-utveckling/oppen-tillgang-och-bibsamkonsortiet/open-access-and-bibsam-consortium/open-access/creative-commons-faq-for-researchers.html
% https://kib.ki.se/en/publish-analyse/publish-your-article-open-access/open-licence-your-publication-cc
\begin{comment}
\usepackage[
    type={CC},
    %modifier={by-nc-nd},
    %version={4.0},
    modifier={by-nc},
    imagemodifier={-eu-88x31},  % to get Euro symbol rather than Dollar sign
    hyphenation={RaggedRight},
    version={4.0},
    %modifier={zero},
    %version={1.0},
]{doclicense}
\end{comment}

\begin{document}
%\selectlanguage{swedish}
%
\selectlanguage{english}

%%% Set the numbering for the title page to a numbering series not in the preface or body
\pagenumbering{alph}
%\kthcover
\clearpage\thispagestyle{empty}\mbox{} % empty back of front cover
\titlepage

% If you do not want to have a bookinfo page, comment out the line saying \bookinfopage and add a \cleardoublepage
% If you want a bookinfo page: you will get a copyright notice, unless you have used the doclicense package in which case you will get a Creative Commons license. To include the doclicense package, uncomment the configuration of this package above and configure it with your choice of license.
\bookinfopage

% Frontmatter includes the abstracts and table-of-contents
\frontmatter
\setcounter{page}{1}
\begin{abstract}
% The first abstract should be in the language of the thesis.
% Abstract fungerar på svenska också.
  \markboth{\abstractname}{}
\begin{scontents}[store-env=lang]
eng
\end{scontents}
%%% The contents of the abstract (between the begin and end of scontents) will be saved in LaTeX format
%%% and output on the page(s) at the end of the thesis with information for DiVA facilitating the correct
%%% entry of the meta data for your thesis.
%%% These page(s) will be removed before the thesis is inserted into DiVA.
% \engExpl{All theses at KTH are \textbf{required} to have an abstract in both \textit{English} and \textit{Swedish}.}
% \engExpl{Exchange students may want to include one or more abstracts in the language(s) used in their home institutions to avoid the need to write another thesis when returning to their home institution.}

% \generalExpl{Keep in mind that most of your potential readers are only going to read your \texttt{title} and \texttt{abstract}. This is why the abstract must give them enough information so that they can decide if this document is relevant to them or not. Otherwise, the likely default choice is to ignore the rest of your document.\\
% An abstract should stand on its own, i.e., no citations, cross-references to the body of the document, acronyms must be spelled out, \ldots .\\Write this early and revise as necessary. This will help keep you focused on what you are trying to do.}

% \begin{scontents}[store-env=abstracts,print-env=true]
% \generalExpl{Enter your abstract here!}
% Write an abstract that is about 250 and 350 words (1/2 A4-page)  with the following components:
% key parts of the abstract
% \begin{itemize}
%  \item What is the topic area? (optional) Introduces the subject area for % the project.
%   \item Short problem statement
%   \item Why was this problem worth a Bachelor's/Master’s thesis project? (\ie, why is the problem both significant and of a suitable degree of difficulty for a Bachelor's/Master’s thesis project? Why has no one else solved it yet?)
%  \item How did you solve the problem? What was your method/insight?
%  \item Results/Conclusions/Consequences/Impact: What are your key results/\linebreak[4]conclusions? What will others do based on your results? What can be done now that you have finished - that could not be done before your thesis project was completed?
% \end{itemize}

%\end{scontents}
Scheduling staff in the healthcare and social care sector is often difficult due to the need to e.g. balance employee preferences, work-hour limits, and unpredictable absences like sick leave. Many organizations still rely on manual scheduling, which is not only time-consuming, but also leaves room for mistakes and often fails to handle last-minute changes. This can lead to poor coverage, staff overload, and schedules that don’t match what employees can or want to work.

The aim of this project is to create a scheduling system that permits human control while automating shift planning. Without sacrificing flexibility or oversight, the goal is to better reflect employee needs, save time, and prevent conflicts. Additionally, with this project, we want to address the scientific and practical challenge of designing and implementing a smart scheduling system that supports automated shift allocation while maintaining managerial control through manual adjustment features. The goal is to reduce administrative workload, improve scheduling accuracy, and respect employee availability and preferences—thus increasing job satisfaction and minimizing burnout risk.

The already-existing Timefold project, which uses artificial intelligence to algorithmically find the best solutions to planning problems with multiple constraints, will be used to create an efficient system. These limitations fall into two categories: "soft" constraints, like employee preferences, and "hard" constraints, like laws or regulations. Both hard and soft constraints can be managed with the aid of Timefold, which is a constraint solver. With the help of drag-and-drop functionality, shift summaries, and simple editing tools, the system's web interface—which was developed with Lovable and Vite—enables schedulers to assign, edit, and manage shifts quickly. Supabase, an open-source online PostgreSQL database that is linked to the backend, was utilized for backend functions. The Postman Agent tool is also used to test the API's dependability before the backend and full-stack deployment are fully integrated. \todo{Ska jag lägga in mer saker här?}

% To help us in building an effective system, we will use an already implemented project, Timefold, which uses artificial intelligence to algorithmically find an optimal solution to a problem with multiple constraints. These constraints are divided into “hard” constraints (eng: \textit{hard constraint}), and “soft” constraints (eng: \textit{soft constraint}). Such hard constraints could be laws, while soft constraints could be preferences of the employees.

% In order to build the system, we will be utilizing Timefold—a constraint solver tool with the capability of taking care of hard constraints, like the number of working hours and resting periods, and soft constraints, like employee shift preferences. 
% With the help of drag-and-drop functionality of shifts, workload summaries, and shift editing, someone who does the scheduling can quickly assign and modify shifts on the web interface. The web interface was created with Lovable and Vite. We use Supabase, an online postgres database that is open-source, and connect it to the backend. Before the complete integration of the backend via Supabase and full-stack deployment, the API is tested for reliability with the Postman Agent tool.

Real users, particularly those that make schedules at care facilities, will be allowed to test the system, interface quality, and the accuracy of the scheduling, in order to gather some input and improve the scheduling system. \textbf{The performance of the system will be evaluated as there's an increase in the amount of workers or shifts.} The end result of the project will be a functional prototype that is contemporary and modern, and that can meet modern-day real-world demands. Furthermore, it also establishes the foundation for a system on which more features can be added and broadens the scope of possibilities for modern scheduling systems. 

% \todo{Kopia med paragrafen innan, vilken version passar bättre?}
% The outcome of the thesis project is a functioning prototype of a scheduling system that meets critical operational and usability demands. It lays a technical foundation for scalable deployment and opens opportunities for further enhancements such as real-time availability tracking and employee feedback integration. The system contributes to the ongoing digital transformation in care-related fields by improving scheduling processes that previously relied heavily on manual input.

% The following commands can be used: \textbackslash eg, \textbackslash Eg, \textbackslash ie, \textbackslash Ie, \textbackslash etc, and \textbackslash etal: \eg, \Eg, \ie, \Ie, \etc, and \etal.

% The following commands for numbering with lowercase Roman numerals: \textbackslash first, \textbackslash Second, \textbackslash third, \textbackslash fourth, \textbackslash fifth, \textbackslash sixth, \textbackslash seventh, and \textbackslash eighth: \first, \Second, \third, \fourth, \fifth, \sixth, \seventh, and \eighth. Note that the second case is set with a capital 'S' to avoid conflicts with the use of second of as a unit in the \texttt{siunitx} package.

% Equations using \textbackslash( xxxx \textbackslash) or \textbackslash[ xxxx \textbackslash] can be used in the abstract. For example: \( (C_5O_2H_8)_n \)
% or \[ \int_{a}^{b} x^2 \,dx \]
% Note that you \textbf{cannot} use an equation between dollar signs.

% Even LaTeX comments can be handled, for example: \% comment.
% Note that one can include percentages, such as: 51\% or \SI{51}{\percent}.

\subsection*{Keywords}
\begin{scontents}[store-env=keywords,print-env=true]

% If you set the EnglishKeywords earlier, you can retrieve them with:
\InsertKeywords{english}
% If you did not set the EnglishKeywords earlier then simply enter the keywords here:
% comma separate keywords, such as: Canvas Learning Management System, Docker containers, Performance tuning
\end{scontents}

% \engExpl{\textbf{Choosing good keywords can help others to locate your paper, thesis, dissertation, \ldots and related work.}}
% Choose the most specific keyword from those used in your domain, see for example: the ACM Computing Classification System ({\small \url{https://www.acm.org/publications/computing-classification-system/how-to-use})},
% the IEEE Taxonomy ({\small \url{https://www.ieee.org/publications/services/thesaurus-thank-you.html}}), PhySH (Physics Subject Headings)\linebreak[4] ({\small \url{https://physh.aps.org/}}), \ldots or keyword selection tools such as the  National Library of Medicine's Medical Subject Headings (MeSH)  ({\small \url{https://www.nlm.nih.gov/mesh/authors.html}}) or Google's Keyword Tool ({\small \url{https://keywordtool.io/}})\\

% \textbf{Formatting the keywords}:
% \begin{itemize}
%   \item The first letter of a keyword should be set with a capital letter and proper names should be capitalized as usual.
%   \item Spell out acronyms and abbreviations.
%  \item Avoid "stop words" - as they generally carry little or no information.
%  \item List your keywords separated by commas (",").
% \end{itemize}    
% Since you should have both English and Swedish keywords - you might think of ordering them in corresponding order (\ie, so that the n\textsuperscript{th} word in each list correspond) - this makes it easier to mechanically find matching keywords.
\end{abstract}
\cleardoublepage
\babelpolyLangStart{swedish}
\begin{abstract}
    \markboth{\abstractname}{}
\begin{scontents}[store-env=lang]
swe
\end{scontents}

Att schemalägga personal inom hälso- och sjukvårdssektorn är ofta svårt på grund av behovet av att bland annat balansera medarbetarnas preferenser, arbetstidsbegränsningar och oförutsägbar frånvaro som sjukfrånvaro. Många organisationer förlitar sig fortfarande på manuell schemaläggning, vilket inte bara är tidskrävande utan också lämnar utrymme för misstag och ofta misslyckas med att hantera förändringar i sista minuten. Detta kan leda till dålig täckning, överbelastning av personal och scheman som inte matchar vad medarbetarna kan eller vill arbeta. Syftet med detta projekt är att skapa ett schemaläggningssystem som möjliggör mänsklig kontroll, samtidigt som skiftplaneringen automatiseras. Utan att offra flexibilitet eller tillsyn är målet att bättre återspegla medarbetarnas behov, spara tid och förhindra konflikter. Dessutom vill vi med detta projekt ta itu med den vetenskapliga och praktiska utmaningen att utforma och implementera ett smart schemaläggningssystem som stöder automatiserad skiftallokering samtidigt som ledningens kontroll bibehålls genom manuella justeringsfunktioner. Målet är att minska den administrativa arbetsbelastningen, förbättra schemaläggningens noggrannhet och respektera medarbetarnas tillgänglighet och preferenser – vilket ökar arbetstillfredsställelsen och minimerar risken för utbrändhet. 

Det redan existerande Timefold-projektet, som använder artificiell intelligens för att algoritmiskt hitta de bästa lösningarna på planeringsproblem med flera begränsningar, kommer att användas för att skapa ett effektivt system. Dessa begränsningar faller inom två kategorier: "mjuka" begränsningar, som anställdas preferenser, och "hårda" begränsningar, som lagar eller förordningar. Både hårda och mjuka begränsningar kan hanteras med hjälp av Timefold, som är en begränsningslösare. Med hjälp av dra-och-släpp-funktionalitet, skiftsammanfattningar och enkla redigeringsverktyg gör systemets webbgränssnitt – som utvecklades med Lovable och Vite – det möjligt för schemaläggare att snabbt tilldela, redigera och hantera skift. Supabase, en öppen källkods-online PostgreSQL-databas som är länkad till backend, användes för backend-funktioner. Postman Agent-verktyget används också för att testa API:ets tillförlitlighet innan backend- och fullstack-distributionen är helt integrerad. \todo{Ska jag lägga in mer saker här?} Riktiga användare, särskilt de som gör scheman på vårdinrättningar, kommer att få testa systemet, gränssnittskvaliteten och schemaläggningens noggrannhet för att samla in lite input och förbättra schemaläggningssystemet. \textbf{Systemets prestanda kommer att utvärderas allt eftersom antalet arbetare eller skift ökar.} Slutresultatet av projektet kommer att vara en funktionell prototyp som är modern och modern, och som kan möta dagens verkliga krav. Dessutom lägger det grunden för ett system där fler funktioner kan läggas till och breddar möjligheterna för moderna schemaläggningssystem.

% Schemaläggningen av personal inom vård- och omsorgssektorn är i dagsläget en utmaning då det finns ett behov att exempelvis balansera preferenser hos anställda, arbetstidslagen, och frånvaro som kan vara oförutsägbar som sjukskrivning. Organisationer sätter sin tillit till en tidskrävande process som manuell schemaläggning, som därutöver sin tidskrav också skapar utrymmet för misstag och kan ofta också misslyckas med att kunna hantera ändringar i sista minuten. Andra utmaningar kan vara sådant som avsaknaden av tillräcklig bemanning, överbelastning av personal, och scheman som inte matchas med de anställda som kan, eller vill, arbeta.

%Med detta projekt åsyftas skapandet av ett schemaläggningssystem som kan möjliggöra mänsklig tillsyn, samtidigt som en andel av skiftplaneringen kan automatiseras. Detta ska kunna göras utan att man ska behöva offra flexibiliteten eller kontroll, med målet att bättre kunna återspegla de behoven som finns hos medarbetare, samt kunna spara tid och förebygga att det uppstår olika konflikter. Vidare vill vi ta itu med de praktiska och vetenskapliga utmaningar som kommer med att designa och implementera ett schemaläggningssystem som är smart och stödjer automatiserad tilldelning av skift, medan ledningskontroll bibehålls via manuella justeringsfunktioner. Som mål vill vi se om det går att minska administrativ arbetsbelastning, förbättrandet av nogrannheten i schemaläggning, och ta hänsyn till tillgängligheten hos personal samt deras preferenser. Utfallet hoppas vi kunna bli en ökad arbetstillfredsställelse samt minimering av risk för utbrändhet. 

%Dra-och-släpp funktioner för skift, arbetsbelastningssammanfattningar och skiftredigering, kan underlätta schemaläggningen genom den kvicka tilldelandet och ändring av skift i webbgränssnittet. I detta projekt skapas webbgränssnittet genom Lovable och Vite. Supabase, en öppen källkod postgres online-databas, används för att ansluta den till backend. Innan full-stack distributionen har fullbordats, och den fullständiga integrationen har gjorts av backend via Supabase, testas API:et för tillförlitligheten med ett verktyg som Postman Agent.

%Verkliga användare, i synnerhet de som gör scheman på vårdinrättningar, kommer få möjligheten att testa systemet, kvaliteten på gränssnittet, och noggrannheten hos schemaläggningen, i syftet att samla ihop input och förbättra schemaläggningssystemet. \textbf{Prestandan hos systemet kommer utvärderas då antalet arbetare eller skift ökar.} Slutresultatet av projektet blir en funktionell prototyp som är modern och som kan möta verkliga krav idag. Dessutom lägger projektet grunden för ett system där flera funktioner kan tilläggas, och ger en bredare omfattning av möjligheter för ett schemaläggningssytem som är mer modern.

%\todo{Kopia med paragrafen innan, vilken version passar bättre?}
%Resultatet av examensarbetet är en fungerande prototyp av ett schemaläggningssystem som möter kritiska drift- och användbarhetskrav. Det lägger en teknisk grund för skalbar distribution och öppnar möjligheter för ytterligare förbättringar som spårning av tillgänglighet i realtid och integration av feedback från anställda. Systemet bidrar till den pågående digitala transformationen inom vårdrelaterade områden genom att förbättra schemaläggningsprocesser som tidigare var mycket beroende av manuell inmatning.

%\warningExpl{Inside the following scontents environment, you cannot use a \textbackslash include{filename} as it will not end up in the for diva information. Additionally, you should not use a straight double quote character in the abstracts or keywords, use two single quote characters instead.}
%\begin{scontents}[store-env=abstracts,print-env=true]
%\generalExpl{Enter your Swedish abstract or summary here!}
%\sweExpl{Alla avhandlingar vid KTH \textbf{måste ha} ett abstrakt på både \textit{engelska} och \textit{svenska}.\\
%Om du skriver din avhandling på svenska ska detta göras först (och placera det som det första abstraktet) - och du bör revidera det vid behov.}

%\engExpl{If you are writing your thesis in English, you can leave this until the draft version that goes to your opponent for the written opposition. In this way, you can provide the English and Swedish abstract/summary information that can be used in the announcement for your oral presentation.\\If you are writing your thesis in English, then this section can be a summary targeted at a more general reader. However, if you are writing your thesis in Swedish, then the reverse is true – your abstract should be for your target audience, while an English summary can be written targeted at a more general audience.\\This means that the English abstract and Swedish sammnfattning  
%or Swedish abstract and English summary need not be literal translations of each other.}

%\warningExpl{Do not use the \textbackslash glspl\{\} command in an abstract that is not in English, as my programs do not know how to generate plurals in other languages. Instead, you will need to spell these terms out or give the proper plural form. In fact, it is a good idea not to use the glossary commands at all in an abstract/summary in a language other than the language used in the \texttt{acronyms.tex file} - since the glossary package does \textbf{not} support use of more than one language.}

%\engExpl{The abstract in the language used for the thesis should be the first abstract, while the Summary/Sammanfattning in the other language can follow}
%\end{scontents}
\subsection*{Nyckelord}
\begin{scontents}[store-env=keywords,print-env=true]
% SwedishKeywords were set earlier, hence we can use alternative 2
\InsertKeywords{swedish}
\end{scontents}
%\sweExpl{Nyckelord som beskriver innehållet i uppsatsen eller rapporten}
\end{abstract}
\babelpolyLangStop{swedish}

\cleardoublepage

\section*{Acknowledgments}
\markboth{Acknowledgments}{}
\sweExpl{Författarnas tack}

\engExpl{It is nice to acknowledge the people that have helped you. It is
  also necessary to acknowledge any special permissions that you have gotten –
  for example, getting permission from the copyright owner to reproduce a
  figure. In this case, you should acknowledge them and this permission here
  and in the figure’s caption. \\
  Note: If you do \textbf{not} have the copyright owner’s permission, then you \textbf{cannot} use any copyrighted figures/tables/\ldots . Unless stated otherwise all figures/tables/\ldots are generally copyrighted.
}
\sweExpl{I detta kapitel kan du ev nämna något om
  din bakgrund om det påverkar rapporten på något sätt. Har du t ex inte
  möjlighet att skriva perfekt svenska för att du är nyanländ till landet kan
  det vara på sin plats att nämna detta här. OBS, detta får dock inte vara en
  ursäkt för att lämna in en rapport med undermåligt språk, undermålig grammatik och
  stavning (t ex får fel som en automatisk stavningskontroll och
  grammatikkontroll kan upptäcka inte förekomma)\\
En dualism som måste hanteras i hela rapporten och projektet
}

I would like to thank xxxx for having yyyy. Or in the case of two authors:\\
We would like to thank xxxx for having yyyy.

\acknowlegmentssignature

\fancypagestyle{plain}{}
\renewcommand{\chaptermark}[1]{ \markboth{#1}{}} 
\tableofcontents
  \markboth{\contentsname}{}

\cleardoublepage
\listoffigures

\cleardoublepage

\listoftables
\cleardoublepage
\lstlistoflistings\engExpl{If you have listings in your thesis. If not, then remove this preface page.}
\cleardoublepage
% Align the text expansion of the glossary entries
\newglossarystyle{mylong}{%
  \setglossarystyle{long}%
  \renewenvironment{theglossary}%
     {\begin{longtable}[l]{@{}p{\dimexpr 2cm-\tabcolsep}p{0.8\hsize}}}% <-- change the value here
     {\end{longtable}}%
 }
%\glsaddall
%\printglossaries[type=\acronymtype, title={List of acronyms}]
\printglossary[style=mylong, type=\acronymtype, title={List of acronyms and abbreviations}]
%\printglossary[type=\acronymtype, title={List of acronyms and abbreviations}]

%\printnoidxglossary[style=mylong, title={List of acronyms and abbreviations}]
\engExpl{The list of acronyms and abbreviations should be in alphabetical order based on the spelling of the acronym or abbreviation.
}

% if the nomenclature option was specified, then include the nomenclature page(s)
% \ifnomenclature
%    \cleardoublepage
    % Output the nomenclature list
%    \printnomenclature
% \fi

%% The following label is essential to know the page number of the last page of the preface
%% It is used to compute the data for the "For DIVA" pages
\label{pg:lastPageofPreface}
% Mainmatter is where the actual contents of the thesis goes
\mainmatter
\glsresetall
\renewcommand{\chaptermark}[1]{\markboth{#1}{}}
\selectlanguage{english}
\chapter{Introduction}
\label{ch:introduction}

En \gls{UI} är definierad som ett användargränssnitt, \gls{UX}, \gls{API}, \gls{ORM}, \gls{DB}, \gls{SQL}, \gls{REST}.

% \sweExpl{Ofta kommer problemet och problemägaren från industrin där man önskar en specifik lösning på ett specifikt problem. Detta är ofta ”för smalt” definierat och ger ofta en ”för smal” lösning för att resultatet skall vara intressant ur ett mer allmänt ingenjörsperspektiv och med ”nya” erfarenheter som resultat. Fundera tillsammans med projektets intressenter (student, problemägare och akademi) hur man skulle kunna använda det aktuella problemet/förslaget för att undersöka någon ingenjörsaspekt och vars resultat kan ge ny eller kompletterande erfarenhet till ingenjörssamfundet och vetenskapen.\\slöser man en del eller hela delen av det ursprungliga problemet.\\Erfarenheten kommer ur en frågeställning som man i examensarbetet försöker besvara med tidigare och andras erfarenhet, egna eller modifierade metoder som ger ett resultat vilket kan användas för att diskutera ett svar på undersökningsfrågan.\\Detta stycke skall alltså, förutom det ursprungliga ”smala” problemet, innehålla  vad som skall undersökas för att skapa ny ingenjörserfarenhet och/eller vetenskap.}

% \engExpl{The first paragraph after a heading is not indented, all of the subsequent paragraphs have their first line indented.}
  
% This chapter describes the specific problem that this thesis addresses, the context of the problem, the goals of this thesis project, and outlines the structure of the thesis.\\

% \generalExpl{Give a general introduction to the area. (Remember to use appropriate references in this and all other sections.)}

% One can use either biblatex or bibtex - set as the option for the document at the top of this file
\ifbiblatex
\engExpl{We use the \emph{biblatex} package to handle our references.  We
use the command \texttt{parencite} to get a reference in parenthesis, like
this \textbackslash parencite\{heisenberg2015\} resulting in \parencite{heisenberg2015}.  It is also possible to include the author as part of the sentence using \texttt{textcite}, like talking about the work of \textbackslash textcite\{einstein2016\} resulting in \textcite{einstein2016}.\\
This also means that you have to change the include files to include biblatex and change the way that the \texttt{reference.bib} file is included.}
\else
% \engExpl{We use the \emph{bibtex} package to handle our references.  We, therefore,
% use the command \textbackslash cite\{farshin\_make\_2019\}. For example, Farshin, \etal described how to improve LLC
% cache performance in \cite{farshin_make_2019} in the context of links running
% at \qty{200}{Gbps}.}
\fi

% Hälso- och socialvårdssektorn ställs inför ökade krav på effektivitet och arbetsmiljöförbättringar. Personalplanering och schemaläggning är centrala för att uppnå detta, men nuvarande manuella system är ofta otillräckliga. \textbf{Från ett ingenjörsperspektiv} erbjuder denna komplexitet en möjlighet att undersöka hur tekniska lösningar kan förbättra både flexibilitet och regelefterlevnad i ett mycket dynamiskt och reglerat sammanhang.

The health and social care sector is facing increased demands for efficiency and work environment improvements. Human resource planning and scheduling are central to achieving this, and in this project we want to investigate how manual scheduling can be facilitated by automation. 

% Många befintliga schemaläggningssystem lider av brister som leder till manuella, resurskrävande processer. Automatiserade lösningar kan potentiellt minska den administrativa bördan, förbättra arbetsmiljön för personalen och i förlängningen bidra till en högre vårdkvalitet genom att minimera utmattningsrisker. 

Many existing scheduling systems suffer from shortcomings that lead to manual, resource-intensive processes. Automated solutions have the potential to reduce the administrative burden, improve the working environment for staff, and in the long run contribute to a higher quality of care by minimizing the risk of fatigue. \cite{bergstedt_schemagi_nodate} \cite{ronnberg_automating_2010} 

\textbf{From an engineering perspective}, this complexity offers an opportunity to explore how technical solutions can improve both flexibility and regulatory compliance in a highly dynamic and regulated context. \cite{socialstyrelsen_2025_planeringsstod} 

% Detta examensarbete syftar till att undersöka hur ett schemaläggningssystem som är automatiserat kan möta kraven. Som utgångspunkt har vi satt för projektet ett fokus på att kombinera en modern "AI constraint solver", Timefold, för att få ut en optimerad lösning på schemaläggningsproblemet medan lagstadgade krav respekteras. Till det kommer vi att använda ett gränssnitt som är användarvänligt för schemaläggare. 


This thesis aims to investigate how an automated scheduling system can meet the requirements. As a starting point, we have set a focus for the project on combining a modern "AI constraint solver", Timefold, to obtain an optimized solution to the scheduling problem while respecting legal requirements. Furthermore, we will work on developing a \gls{UI} that is user-friendly for schedulers that gives a nice \gls{UX}.

\newpage

% Hälso- och socialvårdssektorn står idag inför utmaningar när det gäller personalplanering och schemaläggning—en fråga som inte minst är angeläget och relevant för ingenjörer och utvecklare, men även för anställda inom vården och administratörer. I dessa miljöer har administratörer fått ha många bollar i luften när de ska balansera mellan olika konkurrerande krav, bland annat arbetsmiljölagstiftning och kollektivavtal, önskemål och preferenser hos anställda, driftbehov hos verksamheten, och lagstadgade viloperioder.

% För att kunna säkerställa en högkvalitativ patientvård behövs en strömlinjeformad process för personalplanering som tar hänsyn till alla inblandade parters behov. Traditionella sätt att schemalägga präglas av manuella moment med tidskrävande justeringar som kan leda till suboptimala lösningar och hög administrativ belastning. En automatiserad lösning för schemaläggning hade inte bara effektiviserat arbetsflödet för administratörer, men också skapat en förbättring för personalens arbetsmiljö och dämpat risken för utmattning. I förlängningen hade detta också fått en förbättring i vårdkvaliteten när utmattningsrisker har minimerats. 

% En modern schemaläggningslösning bör till exempel kunna hitta en optimal lösning för skiftfördelning, som tar hänsyn till många variabler som tillgänglighet, kompetens, individuella preferenser, samtidigt som lagar och regler om arbetstid och vila efterlevs. En sådan system hade också behövt ha en kapacitet att hantera förändringar som sker oförutsett, t.ex. sjukfrånvaro, snabb identifiering av tillgängliga ersättare, och kunna kommunicera ut lediga pass via SMS eller mobilapplikationer. På detta sätt kan både flexibilitet och regelefterlevnad förbättras.

% \section{Background}
% \label{sec:background}
% \sweExpl{svensk: Bakgrund}

% \generalExpl{Present the background for the area. Set the context for your project – so that your reader can understand both your project and this thesis. (Give detailed background information in Chapter 2 - together with related work.)
% Sometimes it is useful to insert a system diagram here so that the reader
% knows what are the different elements and their relationship to each
% other. This also introduces the names/terms/… that you are going to use
% throughout your thesis (be consistent). This figure will also help you later
% delimit what you are going to do and what others have done or will do.}

% As one can find in RFC 1235\,\cite{ioannidis_coherent_1991} multicast is useful for xxxx. 

% Inom vård- och omsorgssektorn finns det betydande utmaningar kopplade till schemaläggning. Att manuellt balansera alla inblandades behov, såsom lagkrav, individuella önskemål, avtalade arbetstider och dygnsvila, är en komplex och tidskrävande uppgift. Detta intygas av de samtalen som har gjorts med vårdanställda, anhöriga som arbetar inom vårdsektorn, samt schemaläggare. 

% I detta examensarbete har vi undersökt deras upplevelser kring schemaläggningen inom vården. Under luppen har vi satt effektiviteten som finns i arbetet kring schemaläggningen, och i vilken kapacitet förmågan finns för att möta individuella behov samt önskemål, och eventuella förbättringar som kan göras inom de befintliga schemaläggningssystemen utifrån ovannämndas önskemål. 

% En del utmaningar har identifierats inom befintliga schemaläggningssystem som många har upplevt som otillräckligt flexibla för att kunna hantera den komplexa verklighet som existerar inom vården. De som jobbar med schemaläggningen tampas med att kunna balansera mellan behoven hos verksamheten, och önskemål hos personal, medan lagstadgade krav kring vila och arbetstider ska uppfyllas. 

% Den resursintensitet och tidskrav som förorsakas av manuell schemaläggning tar i själva verket bort uppmärksamheten från andra viktiga arbetsuppgifter. Problemen blir särskilt påtagliga vid plötsliga förändringar och önskemål, som sjukfrånvaro, att anställda vill åka på semester, eller akuta personalbehov i allmänhet. Den som sköter schemaläggningen behöver då hitta en ersättare i en handvändning. 

% Schemaläggare upplever en svårighet att hålla alla önskemål eller preferenser om arbetspass och tider i huvudet, då detta är något som de anställda vill på något sätt ska synas när de blir tilldelade sina skift. Sedan har nuvarande system inte en funktion som, på ett automatiskt sätt, visar antalet timmar de anställda har kvar att arbeta. Detta medför att en schemaläggare manuellt behöver hålla ordning på detta. Skulle en automatiserad funktion implementeras för att underlätta spårningen, hade detta förenklat processen en hel del.

% Schemaläggaren som vi har intervjuat har rest några problem de har haft med det nuvarande systemet, och känner att den upplevs som knöligt och saknar intuitiva funktioner, t.ex. drag-and-drop och kortkommandon. Ett användarvänligare gränssnitt skulle kunna förbättra arbetsflödet och minska antalet klick och manuella moment.

% Det finns i dagsläget vissa lösningar som saknas i befintliga schemaläggningssystem, och som skulle kunna implementeras. \todo{Kanske flytta stycket "En lösning skulle, …" till resultat eller en annan sektion?}En lösning skulle, som exempel, vara att kunna integrera automatiska utskickningar av SMS-förfrågningar till tillgänglig personal vid kort varsel. Detta måste dock ske med hänsyn till individuella önskemål, lagkrav och dygnsvila.

% Vidare vill vi med detta examensarbete undersöka möjligheterna till skapandet av ett automatiserat—och potentiellt skalbart—schemaläggningssystem med hjälp av dagens teknologi och artificiell intelligens. Vi fokuserar även på användargränssnittet och användarvänligheten för att effektivisera arbetet för schemaläggningspersonal.

% \todo{Flytta texten under till en annan sektion}
% Till vår hjälp kommer vi använda en redan implementerad projekt, Timefold, som använder artificiell intelligens för att algoritmiskt hitta en optimal lösning till ett problem med flera villkor. Dessa villkor är indelade i ”hårda” villkor (eng: \textit{hard constraint}), och ”mjuka” villkor (eng: \textit{soft constraint}). Sådana hårda villkor skulle kunna vara lagar, medan mjuka villkor skulle kunna vara preferenser hos de anställda.

(\textbf{<Skriva om front end lite våra idéer…>})
(\textbf{När det gäller ramverk så föll valet på att använda Next.js, eftersom…})
(\textbf{<Vi kopplar texten om frontend till texten efter ”:” :>}) 

För att uppnå en schemaläggningsystem som är lättanvänd och flexibel kommer det erbjudas funktionaliteter som drag-and-drop, och experimentera med användargränssnittet. Löpande under utvecklingens gång kommer användargränssnittet att utvärderas för att utforska hur användarupplevelsen kan förbättras.


\section{Problem}
\label{sec:problem}
% \sweExpl{svensk: Problemdefinition eller Frågeställning\\
% Lyft fram det ursprungliga problemet om det finns något och definiera därefter
% den ingenjörsmässiga erfarenheten eller/och vetenskapen som kan komma ur
% projektet. }


% Schemaläggning inom vård och omsorg kräver att man tar hänsyn till flera faktorer: lagkrav, kollektivavtal, personalens önskemål och verksamhetens behov. Dessa måste balanseras, vilket idag sker genom arbetsintensiva manuella processer. I samtal med schemaläggare och personal framkommer att dagens system är ineffektiva, saknar flexibilitet och intuitiva funktioner.

% Research Question
\subsection{Original problem and definition}
\label{sec:researchQuestion}
The situation in the healthcare sector requires that scheduling takes into account several different things, such as legislation on the work environment, individual wishes from employees, OR the operational needs of the care facility. The question of how the balance should be handled is often done through routines that are both labor-intensive and manual. The interviews and conversations that have been conducted with scheduling staff, as well as relatives in the capacity of being healthcare staff, have given us a glimpse into the rigidity of modern scheduling systems, with a flexibility that otherwise leaves much to be desired. Further, having user-friendly functions that contribute to a good \gls{UX} has been stated to be an effective addition, and to have real-time updates of the working time that remains. \cite{ohlund2025}


\subsection{Scientific and engineering issues}
From the point of view of engineering, the area within which we are working in is interesting for exploring how employee scheduling software can help support human decision-making while making certain that there is a compliance with legal regulations.   It also provides an opportunity to explore the ways in which constraint solving tools and user-friendly interfaces can streamline the scheduling process.

\section{Purpose}
\sweExpl{Skilj på syfte och mål! Syfte är att förändra något till det bättre. I examensarbetet finns ofta två aspekter på detta. Dels vill problemägaren (företaget) få sitt problem löst till det bättre men akademin och ingenjörssamfundet vill också få nya erfarenheter och vetskap. Beskriv ett syfte som tillfredställer båda dessa aspekter.\\
Det finns även ett syfte till som kan vara värt att beakta och det är att du som student skall ta examen och att du måste bevisa, i ditt examensarbete, att du uppfyller examensmålen. Dessa mål sammanfaller med kursmålen för examensarbetskursen. 
}
\generalExpl{State the purpose of your thesis and the purpose of your degree project.\\
Describe who benefits and how they benefit if you achieve your goals. Include anticipated ethical, sustainability, social issues, etc. related to your project. (Return to these in your reflections in Section~\ref{sec:reflections}.)}

% For the purpose of this project, we want to make a contribution in both practical and academic terms by making an effort to develop a system for scheduling that streamlines the planning of shifts for employees. With this approach, we are looking to better reflect what the employees have as their needs and preferences, cut the administrative workload, and avoiding the conflict of scheduling – something that in the long run could help with the job satisfaction and lessen the risk of burnout. The scheduling system that we have worked on, helps offer a concrete solution to a real-world problem in the domains of health and social care, where it is a laborious and error-prone process to do the scheduling. Prospective software developers and engineers interested in learning how to implement a full-stack scheduling system—one that performs scheduling effectively and efficiently, and offers a reasonably good user interface and user experience—will benefit from this project.

We aim to develop a scheduling system that streamlines shift planning for employees by reflecting their needs and preferences, reducing administrative workload, and avoiding scheduling conflicts—ultimately improving job satisfaction and reducing burnout. Our system provides a concrete solution to a real-world problem in health and social care, where scheduling is often laborious and error-prone. Prospective software developers and engineers can benefit from this project by learning how to implement a full-stack scheduling system that is both efficient and user-friendly.

% The purpose of this project is to develop a scheduling system that can streamline the scheduling process while leaving room for some human oversight. This should also be done without compromising control or flexibility in order to better reflect the needs of employees, save time, and avoid conflicts from arising. 

% People who work in the health and social care industry, as well as other prospective software developers and engineers interested in learning how to implement a full-stack scheduling system—one that performs scheduling effectively and efficiently, and offers a reasonably good user interface and user experience—will benefit from this project.


% Med detta projekt åsyftas skapandet av ett schemaläggningssystem som kan möjliggöra mänsklig tillsyn, samtidigt som en stor andel av skiftplaneringen kan automatiseras. Detta ska kunna göras utan att man ska behöva offra flexibiliteten eller kontroll, med målet att bättre kunna återspegla de behoven som finns hos medarbetare, samt kunna spara tid och förebygga att det uppstår olika konflikter. Vidare vill vi ta itu med de praktiska och vetenskapliga utmaningar som kommer med att designa och implementera ett schemaläggningssystem som är smart och stödjer automatiserad tilldelning av skift, medan ledningskontroll bibehålls via manuella justeringsfunktioner. Som mål vill vi se om det går att minska administrativ arbetsbelastning, förbättrandet av noggrannheten i schemaläggning, och ta hänsyn till tillgängligheten hos personal samt deras preferenser. Utfallet hoppas vi kunna bli en ökad arbetstillfredsställelse samt minimering av risk för utbrändhet.


\section{Goals}
%\sweExpl{Mål}
%\sweExpl{Skilj på syfte och mål. Syftet är att åstakomma en förändring i något. Målen är vad som %konkret skall göras för att om möjligt uppnå den önskade förändringen (syfte). }

%\generalExpl{State the goal/goals of this degree project.}

%The goal of this project is XXX. This has been divided into the following three sub-goals:
%\begin{enumerate}
%\item Subgoal 1 \sweExpl{för att tillfredsställa problemägaren – industrin?}
%\item Subgoal 2\sweExpl{för att tillfredsställa ingenjörssamfundet och vetenskapen – akademin) }
%\item Subgoal 3\sweExpl{eventuellt, för att uppfylla kursmålen – du som student}
%\end{enumerate}

% \generalExpl{In addition to presenting the goal(s), you might also state what the deliverables and results of the project are.}
The goal of this project is to develop a scheduling system that will combine automation with human oversight, without compromising on the flexibility. To concretize this goal, we break up our goal into sub-goals:

\begin{enumerate}
    \item Develop a system that allows for computerized shift assignment based on constraints such as availability, rest periods, and legal requirements. This will improve scheduling accuracy and lessen the administrative burden.

    \item Examine how a scheduling system can utilize tools such as Timefold, an AI that solves constraints—while allowing manual adjustments. 

    \item Broadening the horizons of possibilities for prospective developers to create similar systems with the use of AI constraint solvers, and to understand how to achieve a relatively good user-friendly scheduling system that respects legal regulations and preferences in the workforce. 
\end{enumerate}


\section{Research Methodology}\sweExpl{Undersökningsmetod}
\sweExpl{Här anger du vilken vilken övergripande undersökningsstrategi eller metod du skall använda för att försöka besvara den akademiska frågeställning och samtidigt lösa det e v ursprungliga problemet. Ofta kan man använda ”lösandet av ursprungsproblemet” som en fallstudie kring en akademisk frågeställning. Du undersöker någon intressant fråga i ”skarpt” läge och samlar resultat och erfarenhet ur detta.\\
Tänk på att företaget ibland måste stå tillbaka i sin önskan och förväntan på projektets resultat till förmån för ny eller kompletterande ingenjörserfarenhet och vetenskap (ditt examensarbete). Det är du som student som bestämmer och löser fördelningen mellan dessa två intressen men se till att alla är informerade. }
\generalExpl{Introduce your choice of methodology/methodologies and method/methods – and the reason why you chose them. Contrast them with and explain why you did not choose other methodologies or methods. (The details of the actual methodology and method you have chosen will be given in Chapter~\ref{ch:methods}. Note that in Chapter~\ref{ch:methods}, the focus could be research strategies, data collection, data analysis, and quality assurance.)\\
In this section you should present your philosophical assumption(s), research method(s), and research approach(es).}

The research methodology that we adopt in this study is empirical, qualitative, and applied. The objective is to create a scheduling system that tackles real-world issues that comes with the territory of doing staff shift planning, particularly in the social and healthcare sector. The process is guided by iteratively developing the system, test it, and then improving it in response to user input and real-world constraints. Aside from conducting interviews with schedulers to learn about their needs and pain points, the research includes a review of current literature on the area that we're working in. 

% The study follows a qualitative research methodology, rooted in interpretivism and guided by an abductive approach. Since we are building a scheduling system to evaluate how well it performs in practice, with real data and specific use cases, this project qualifies as applied research. Our goal is to understand how effectively the system addresses various constraints and user needs, particularly in the context of healthcare and social care. We aim to evaluate whether the system benefits schedulers and supports their work. The research is empirical, as it is grounded in lived experiences of employees and will include user testing. We are iteratively developing the system — building, evaluating, and refining — based on feedback and observations. As part of our method, we conducted a literature study to explore existing tools, platforms, and techniques relevant to shift scheduling. We also examined common shortcomings in current scheduling systems and interviewed employees to gain insights into their needs and challenges, thereby informing our system design and evaluation criteria.


% For our research methodology, we will be employing the design science research (DSR) methodology, where the goal is to solve a real-world problem through the development of an artifact — in this case, a smart scheduling system that includes the backend, frontend, and constraint solver setup. Identifying the problem and motivation is the first step in the design-scientific process. Next, the goals for the solution are defined. Finally, design and development are carried out, followed by demonstration, evaluation, and communication.

\section{Delimitations}
\generalExpl{Describe the boundary/limits of your thesis project and what you are explicitly not going to do. This will help you bound your efforts – as you have clearly defined what is out of the scope of this thesis project. Explain the delimitations. These are all the things that could affect the study if they were examined and included in the degree project.}

% Det här är vad jag själv har skrivit.
% Since the project is not so much concerned with the front-end part, we decided to utilize a tool such as Lovable to streamline the development of the scheduling system, making sure that we don't lose the user friendliness, or obtain a complicated UI with a bad UX. Fortunately, the great thing about the tool is that it doesn't require much coding knowledge, and even more fortunate if you know enough to make adjustments as one wishes.

% Det här är från ChatGPT + Quillbot
We deliberately decided not to concentrate on front-end development for this project in favor of the core scheduling functionality. To this end, we expedited the development of the user interface by employing a low-code tool called Lovable.  This saved us a lot of time on front-end coding and enabled us to create a simple and usable user interface.  Additionally, since the system is not being implemented in an actual organizational context, we chose not to use actual company data.  For testing and demonstration, we instead use dummy data.  Lastly, since this would require infrastructure and testing resources that are outside the purview of this project, we decided not examine the system's scalability features (such as performance under hundreds of users or complicated shift rules).

\section{Outline of the thesis}
% Det här har jag skrivit
% The way in which the report will be outline is that we will be talking about the background broadly in chapter 2, so that the reader can understand what report will be about, and on what previous work and literature it will be built upon. Then, in chapter 3, we will be talking about our methods and methodology, ways in which we have gathered data, et cetera. We will also be talking about how we did what we did, what we have done, and describe our thought processes behind our design, if any. moving onto the result and analysis chapter, we will be talking about our result and findings, where will be addressing some of the research questions that we initially formulated, and based on some predefined metrics evaluate the significance and relevance of our results, as well as talking about whether the goals we initially had in mind have been met. Also, we will be talking about what could be improved in the report and the research broadly, so that others can benefit from this project and understand what the shortcomings were. Wrapping up the report, we will be talking about conclusion under the final chapter, which is where we will establish—in final and succinct terms—what the reader should know if they were on a time crunch and didn't want to read the entire report to get an idea.

% ChatGPT
% In Chapter~\ref{ch:background}, we introduce the background of the study. This includes the broader context of the problem domain, key terminology, and an overview of related work. The purpose is to ensure the reader understands the foundation upon which the project is built. Chapter~\ref{ch:methods} outlines the overall methodology and research approach, including how data was gathered and how the work was structured from a scientific and engineering perspective. Chapter~\ref{ch:whatYouDid} provides a detailed description of the implementation process. We explain what we built, how we built it, and the reasoning behind our design and technical decisions. Chapter~\ref{ch:resultsAndAnalysis} presents the results and evaluates them using predefined metrics. We examine to what extent the project's goals were achieved and address the research questions posed at the outset. In Chapter~\ref{ch:discussion}, we reflect on the implications of our findings, discuss the strengths and weaknesses of our approach, and consider how others might build on or learn from our work. Finally, Chapter~\ref{ch:conclusionsAndFutureWork} offers concluding remarks. We summarize the key takeaways of the project and suggest possible directions for future work and improvements.

% ChatGPT + Quillbot
The presentation of the study's background is in Chapter~\ref{ch:background}, and it comprises an outline of related work, important terminology, and the problem domain's larger context. Making sure the reader comprehends the fundamentals of the project is the aim. The general research approach and methodology, including the data collection process and the work's structure from a scientific and engineering standpoint, are described in Chapter~\ref{ch:methods}. The implementation process is described in detail in Chapter~\ref{ch:whatYouDid}. We describe our design and technical choices, as well as what we built and how we built it. The results are presented and assessed using predetermined metrics in Chapter~\ref{ch:resultsAndAnalysis}. We address the research questions posed at the outset, and assess the degree to which the project's objectives were met. We look into the implications of our findings, talk about the advantages and disadvantages of our methodology, and think about how others could expand upon or benefit from our work in Chapter~\ref{ch:discussion}. Concluding remarks are provided in Chapter~\ref{ch:conclusionsAndFutureWork}, where we provide a summary of the project's main conclusions and offer potential avenues for further research and development.

\cleardoublepage
\chapter{Background}
\label{ch:background}

% \generalExpl{When you do your literature study, you should have a nearly complete Chapters 1 and 2.\\ \\
% You may also find it convenient to introduce the future work section into your report early – so that you can put things that you think about but decide not to do now into this section.\\ \\
% Note that later you can move things between this future work section and what you have done as you may change your mind about what to do now versus what to put off to future work.}

% \generalExpl{What does a reader (another x student -- where x is your study line) need to know to understand your report?
% What have others already done? (This is the “related work”.) Explain what and
% how prior work/prior research will be applied on or used in the degree
% project/work (described in this thesis). Explain why and what is not used in
% the degree project and give valid reasons for rejecting the work/research.}

% This chapter provides basic background information about xxx. Additionally, this chapter describes xxx. The chapter also describes related work xxxx.



% \sweExpl{Vilken viktig litteratur och
%   (forsknings-)artiklar har du studerat inom området (litteraturstudie)? }

% Before you had any kind of software to schedule employees, you had something called a punched card, which is method of recording the amount of hours an employee has worked, as well as doing the actual scheduling. It would be done on stiff piece of paper, and the information would be visible in the presence of holes—or lack thereof—in specific positions that are pre-defined. This was something that IBM was first to popularize the usage of, where they called it "the IBM card." This lead to several improvements, since there was less human error because of fewer recordkeeping by hand. 

\subsection{Some small history}
Before software for tracking the time worked of employees saw the light of the day, organizations made use of mechanical timekeeping techniques. The punched card from the 20th century is a prominent example of this, where it would be used to track the working hours of employees and facilitate the process of scheduling. In order to mark the arrival and departure times, employees would place stiff paper cards into time clocks, and these would in turn stamp the current time onto the card. This is also commonly known as "punching the clock" \cite{cambridge_punchclock}. These cards made it simple to visually interpret work schedules and hours by having holes punched in predetermined places to represent specific data points by having them read by a tabulating machine. \cite{ibm_punched_card} 

% Then came the magnetic tape as a way to store data, with computers being used to access this stored data. Combined, these would render the usage of punched cards obsolete, as they became more affordable as the supply grew. These magnetic tapes would have greater capacity for the storage of data. 

Around the 1960s, magnetic tapes entered the stage with a streamlined way of storing employee data—this is following the period where punch cards were widely in use. Magnetic tapes gave a larger storage capacity compared to the paper-based medium that composed the punched cards, which allowed for a better way to handle data. This change could, for example, be illustrated by releases like IBM's 3850 Mass Storage System, which, with the combination of disk drives and magnetic tapes, enabled a random access to massive amounts of data. \cite{noauthor_undated-hc} These developments have persisted into the modern era, where there has been a rationalized process in the management of the workforce. This can be exemplified by the capacity to store and retrieve large amounts of data, and has enabled the automation of processes like time tracking, processing of payroll, and general administrative work—increasing the accuracy and efficiency of workforce operations. \cite{Angela2024-aw} 
\subsection{Challenges in contemporary scheduling of employees}
Despite all of this abovementioned progress, it is curious how modern employee scheduling processes brims with challenges. As was previously mentioned in the earlier chapter, scheduling process in the workforce involves satisfying many different constraints that leads to a higher level of complexity. The constraints in question are something that is hard to grapple with, as they make the task of establishing a schedule something laborious. The manual process of scheduling employees—such as in the social and healthcare sector—is a widespread activity. It is not out of the ordinary to—in these kinds of environments—witness challenges that emerges from things like sick leave from engendering last-minute changes in the scheduling of employees. \cite{ohlund2025}

\subsection{Types of constraints in scheduling}
It is perhaps a good enough time here to—in more exact terms—explain what the kinds of constraints are that we are talking about. The \textit{hard constraint} is defined as the plurality of real-world rule and limitation that a solution—in scheduling terms—has to respect. Things like not scheduling one person to be present at different places at the same time, or that there is exactly 24 hours in a day, or rules that, for example, has to be adhered to because of e.g. labor laws, or an employee contract. If you break the hard constraints, then a working solution will be unobtainable and removed from reality. The other kind of constraint is the \textit{soft constraint}, and it's relatively more lenient in allowing violations to be broken—in other words, it is non-mandatory to preclude breaking of soft constraints—since it's a subset of feasible solutions and determined by things like the goals of an organization, or the preferences of employees. \cite{timefold_documentation_planning-ai-concepts} \cite{regin2006softalldiff}

It is challenging to create schedules that are both legally compliant and in line with the demands of specific employees—i.e. juggling both hard and soft restraints. Therefore, even though administrative effectiveness has evolved in many areas, the scheduling issue still persists as a bottleneck, demanding the development of more intelligent, constraint-aware algorithms that can adjust to the complexity of real-world labor management. \cite{ohlund2025} \cite{Otero-Caicedo2023-zf} \cite{Ngoo2022-gk}
\subsection{Limitations in current scheduling systems}
Given the above-mentioned complex reality, it is evident that while scheduling tools are currently available, there haven't been any effective methods for managing staff scheduling while simultaneously handling the tension between organizational constraints and employee needs. \cite{researchgate_article} \cite{Burgert2024-uo} \cite{Yasmine2024-gb} \cite{Thomas_2024}

It is not possible to make last-minute necessary modifications with many of the scheduling systems in use today. The need for real-time visibility into important variables, such as the availability of qualified workers or the remaining work-hours of staff, was another pain point brought up by scheduling staff in our conducted interview. This means that several scheduling staff members must remember this information by heart, which introduces the possibility of human mistake, job overload, and unequal shift distribution. \cite{Hur_researchgate} \cite{ohlund2025} \cite{Alaouchiche_researchgate}
\subsection{Need for an adaptive and intelligent scheduling tool}
A contemporary scheduling system must do more than merely adhere to preset templates in order to address these issues.  It ought to be able to update in real time, take into account staff preferences, and abide by legal requirements. It requires intelligent technology and a user-friendly design to accomplish this effectively. \cite{Ernst_researchgate} \cite{ohlund2025}

\subsection{Timefold, PlanningAI and Constraint Satisfaction Programming}
The open-source project OptaPlanner, which Timefold is based on, was co-founded by the OptaPlanner creator. The project was developing for some part of its inception under the guidance of Red Hat, a software company that assists in the provision of open source software products. However, it compelled their hand to move in their own direction when the strategies of Red Hat shifted. In essence, Timefold is an OptaPlanner-forked project. \cite{red_hat} \cite{timefold_optaplanner} 

Timefold assists in creating scheduling solutions that are customized for each individual organization.  Among other things, it can assist in resolving planning issues, such as staff assignment for a schedule.  In order to solve planning issues, a technique known as planning AI and constraint satisfaction programming is employed. This technique finds the best solutions in order to maximize efficiency and resource utilization. \cite{timefold_documentation_introduction} PlanningAI is a subset of artificial intelligence that's dedicated to managing complex scheduling and planning tasks while meeting a variety of constraints. The way in which constraint satisfaction programming works is that it finds a solution for search problems that are combinatorial, and in the context of Timefold, it is based on techniques from operations research. \cite{ROSSI2008181} Timefold describes operation research as the field of research that, with the help of some techniques, looks for the optimal—or near thereof—solutions. This is done in order to advance the decision-making process. Situating the constraint satisfaction programming within this context, it seeks to find a way to satisfy all constraints of a problem. \cite{timefold_planning_ai_concepts}

Timefold enables users to make their own definitions of both soft and hard constraints, such as the preferences of employees, which are desirable but optional, and things like legal working hours, which shouldn't be broken.  Based on these constraints, the system assesses potential solutions in an effort to identify the best solution within the given parameters. \cite{timefold_documentation_timefold_platform} \cite{timefold_documentation_planning-ai-concepts}

\subsection{Why we chose Timefold for this project}
Among the available ecosystem of tools that can be used to resolve the scheduling issues in the workforce, Timefold stood out because of its close match with our technical capabilities. It had a lot of documentation available online, resources, and tutorials that we could utilize without a lot of foreknowledge about intricate and advanced details. Timefold uses Java to define constraints in the code, along with the core solver functionality. This provided us with the possibility to integrate the solver seamlessly with the backend, which was likewise written in Java and used Quarkus and Hibernate. 

According to the announcement of the Timefold project on their blog, the library is "\textit{already faster, lighter and better documented than OptaPlanner}" and seeks to "\textit{make Planning Optimisation easy to use and simple to build}." \cite{timefold_optaplanner} Our own objectives of developing a practical scheduling system align with this approach as well. 

As far as other popular alternatives are concerned, there is for example Google's OR-tools suite, which was also under consideration. While it seemed to have the makings of a promising tool for the project, the main reason we chose to go with Timefold was because of the amount of documentation available online, with a code template that was specific for our implementation of the scheduling system, i.e. employee scheduling. So it was mainly out of convenience's sake that we went with Timefold, and not because of theoretical superiorities.

\section{Major background area 1}
\sweExpl{Viktigt bakgrundsområde 1}
There are xxx characteristics that distinguish yyy from other information and communication technology (ICT) system, as shown in Figure~\ref{fig:lotsofstars}. Table \ref{tab:tablecaracteristics} summarizes these characteristics.


 
\begin{figure}[!ht]
  \begin{center}
    \includegraphics[width=0.5\textwidth]{figures/lots_of_stars.png}
  \end{center}
  \caption{Lots of stars  (Inspired by Figure x.y on page z of [xxx])}
  \label{fig:lotsofstars}
\end{figure}
\sweExpl{Massor av stärnor (Inspirerad av figur x.y på sidan z i [xxx])}


\begin{table}[!ht]
  \begin{center}
    \caption{xxx characteristics}
    \label{tab:tablecaracteristics}
    \begin{tabular}{l|S[table-format=4.6]} % <-- Alignments: 1st column left, 2nd middle, with vertical lines in between
      \textbf{Characteristics} & \textbf{Description}\\
      $\alpha$ & $\beta$ \\
      \hline
      1 & 1110.1\\
      2 & 10.1\\
      3 & 23.113231\\
    \end{tabular}
  \end{center}
\end{table}
\sweExpl{Egenskaper}
\sweExpl{Beskrivning}

\subsection{Subarea 1.1}
Entangled states are an important part of quantum cryptography, but also relevant in other domains. This concept might be relevant for neutrinos, see for example \cite{kim_small-mass_2016}.

\subsection{Subarea 1.1.2}
Computational methods are increasingly used as a third method of carrying out
scientific investigations. For example, computational experiments were used to
find the amount of wear in a polyethylene liner of a hip prosthesis in \cite{maguire_jr_new_2014}.
…

\subsection{Subarea 1.1.2}
Using the nearest data center may improve performance, see \cite{bogdanov_nearest_2015}


\subsection{Link layer Encapsulation}
\label{sec:llencap}

See Figure~\ref{fig:ieee8023-data-packet} which uses the \textsf{bytefield}  \LaTeX\ package. 


\begin{figure}[!ht]
	\centering
\begin{bytefield}{21}
\bitbox[]{7}{} & \bitbox[]{3}{\tiny octets:} & \bitbox[]{4}{\tiny 6} & \bitbox[]{4}{\tiny 6} & \bitbox[]{3}{\tiny 2} & \bitbox[]{5}{\tiny 46 to 1500} & \bitbox[]{3}{\tiny 0 to 46} & \bitbox[]{2}{\tiny 4}\\ 

\bitbox[]{8}{\textbf{ETHERNET \\[-1ex] \tiny{data link-layer}}} & \bitbox[]{2}{} & 

\bitbox{4}{\tiny Destination Address} & \bitbox{4}{\tiny Source Address} & \bitbox{3}{\tiny Length/ Type} & 
\bitbox{5}{\tiny Data Payload} & \bitbox{3}{\tiny Padding} &
\bitbox{2}{\tiny CRC} \\

\bitbox[]{1}{} &\bitbox[]{3}{\tiny octets:} & \bitbox[]{4}{\tiny 7} & \bitbox[]{2}{\tiny 1} & \bitbox[]{0}{$\vdots$ \\[1ex]} & \bitbox[]{16}{} & \bitbox[]{0}{$\vdots$ \\[1ex]} & \bitbox[]{5}{} & \bitbox[]{4}{\tiny Variable}\\

\bitbox[]{4}{\textbf{MAC \\[-1ex] \tiny{packet}}} & \colorbitbox{lightgray}{4}{\tiny Preamble} & \colorbitbox{lightgray}{2}{\tiny SFD} & \colorbitbox{lightgray}{16}{\tiny MAC Client Data} & \colorbitbox{lightgray}{3}{\tiny Padding} &
\colorbitbox{lightgray}{2}{\tiny CRC} & \colorbitbox{lightgray}{4}{\tiny Extension}
\end{bytefield}
     \caption{Ethernet data link layer protocol encapsulated into a IEEE~802.3 MAC packet}
     \label{fig:ieee8023-data-packet}
\end{figure}

\subsection{IP packet headers}
\label{sec:ipheaders}
The data link layer will receive a packet from the IP layer. The layout of
an IPv4 packet is shown in Figure~\ref{fig:ipv4-header}. This should be
contrasted with the IPv6 header shown in Figure~\ref{fig:ipv6-header}.

%
% IPv4 packet header
%
\begin{figure}[!ht]
	\centering
\begin{bytefield}{32}
\bitheader{0-31} \\
\bitbox{4}{\footnotesize{Version}} & \bitbox{4}{IHL} & \bitbox{6}{\tiny{Type of Service}} & \bitbox{2}{{\scriptsize ECN}} \bitbox{16}{Total Length}\\
\bitbox{16}{Identification} & \bitbox{3}{Flags} & \bitbox{13}{Fragment Offset}\\
\bitbox{8}{Time to Live} & \bitbox{8}{Protocol} & \bitbox{16}{Header Checksum}\\
\wordbox{1}{Source Address}\\
\wordbox{1}{Destination Address}\\
\colorbitbox{lightgray}{24}{Options} & \colorbitbox{lightgray}{8}{Padding}
\end{bytefield}
     \caption[IPv4 datagram header]{IPv4 datagram header. Light grey-colored fields are optional.}
    \label{fig:ipv4-header} 
\end{figure}

%
% IPv6 packet header
%
\begin{figure}[!ht]
	\centering
\begin{bytefield}{32}
\bitheader{0-31} \\
\bitbox{4}{\footnotesize{Version}} & \bitbox{8}{Traffic Class} & \bitbox{20}{Flow Label}\\
\bitbox{16}{Payload Length} & \bitbox{8}{Next Header} & \bitbox{8}{Hop Limit}\\
\wordbox{4}{Source Address}\\
\wordbox{4}{Destination Address}\\
\end{bytefield}
     \caption{IPv6 datagram header}
    \label{fig:ipv6-header}
\end{figure}

\subsection{Test for accessibility of formulas}

As can be seen in these equations:
$c=2 \cdot \pi \cdot r$ or \[ \int_{a}^{b} x^2 \,dx \] a chemical formula: $(C_5O_2H_8)_n$
...
\section{Major background area 2}\sweExpl{Viktigt bakgrundsområde 2}
...
\subsection{\glsentryshort{WLAN} Security}% you can't use the \gls() command in a heading - but you can get the short (\glsentryshort) or long version (\glsentryshort) or \glsentrylong or even the text entry (\glsentrytext) and then there is no problem - see https://tex.stackexchange.com/questions/198140/glossaries-and-custom-section-headings-broken

\subsection{Network layer security}
...

\section{Related work area}\sweExpl{Relaterade arbeten}


\subsection{Major related work 1}\sweExpl{Relaterade arbeten 1}
Carrier clouds have been suggested as a way to reduce the delay between the users and the cloud server that is providing them with content. However, there is a question of how to find the available resources in such a carrier cloud. One approach has been to disseminate resource information using an extension to OSPF-TE, see Roozbeh, Sefidcon, and Maguire \cite{roozbeh_resource_2013}.


\subsection{Major related work n}\sweExpl{Relaterade arbeten}

\subsection{Minor related work 1}\sweExpl{Mindre relaterat arbete 1}


…
\subsection{Minor related work n}\sweExpl{Mindre relaterat arbete n}


\section{Summary}\sweExpl{Sammanfattning}
\sweExpl{Det är trevligt om detta kapitel
  avslutas med en sammanfattning. Till exempel kan du inkludera en tabell som
  sammanfattar andras idéer och fördelar och nackdelar med varje - så som
  senare kan du jämföra din lösning till var och en av dessa. Detta kommer
  också att hjälpa dig att definiera de variabler som du kommer att använda
  för din utvärdering.}

\engExpl{It is nice to have this chapter conclude with a summary. For
  example, you can include a table that summarizes other people's ideas and
  benefits and drawbacks with each - so as later you can compare your solution
  to each of them. This will also help you define the variables that you will
  use for your evaluation.}

\cleardoublepage
\chapter{Method or Methods}
\label{ch:methods}
\sweExpl{Metod eller Metodval}
\generalExpl{This chapter is about Engineering-related
  content, Methodologies and Methods.  Use a self-explaining title.\\The
  contents and structure of this chapter will change with your choice of
  methodology and methods.}



\generalExpl{Describe the engineering-related contents (preferably with models) and the research methodology and methods that are used in the degree project.\\
Give a theoretical description of the scientific or engineering methodology  you are going to use and why have you chosen this method. What other methods did you consider and why did you reject them?\\
In this chapter, you describe what engineering-related and scientific skills you are going to apply, such as modeling, analyzing, developing, and evaluating engineering-related and scientific content. The choice of these methods should be appropriate for the problem. Additionally, you should be conscious of aspects relating to society and ethics (if applicable). The choices should also reflect your goals and what you (or someone else) should be able to do as a result of your solution - which could not be done well before you started.}

The purpose of this chapter is to provide an overview of the research method
used in this thesis. Section~\ref{sec:researchProcess} describes the research
process. Section~\ref{sec:researchParadigm} details the research
paradigm. Section~\ref{sec:dataCollection} focuses on the data collection
techniques used for this research. Section~\ref{sec:experimentalDesign}
describes the experimental design. Section~\ref{sec:assessingReliability}
explains the techniques used to evaluate the reliability and validity of the
data collected. Section~\ref{sec:plannedDataAnalysis} describes the method
used for the data analysis. Finally, Section~\ref{sec:evaluationFramework}
describes the framework selected to evaluate xxx.

\sweExpl{Vilka vetenskaplig eller ingenjörs-metodik ska du använda och varför har du valt den här metoden. Vilka andra metoder gjorde du övervägde du och varför du avvisar dem.
Vad är dina mål? (Vad ska du kunna göra som ett resultat av din lösning - vilken inte kan göras i god tid innan du började)
Vad du ska göra? Hur? Varför? Till exempel, om du har implementerat en artefakt vad gjorde du och varför? Hur kommer du utvärdera den.
Syftet med detta kapitel är att ge en översikt över forsknings metod som
används i denna avhandling. Avsnitt~\ref{sec:researchProcess} beskriver forskningsprocessen. Avsnitt~\ref{sec:researchParadigm} beskriver forskningsparadigmen detaljerat. Avsnitt~\ref{sec:dataCollection} fokuserar på datainsamlingstekniker som används för denna forskning. Avsnitt~\ref{sec:experimentalDesign} beskriver experimentell
design. Avsnitt~\ref{sec:assessingReliability} förklarar de tekniker som används för att utvärdera
tillförlitligheten och giltigheten av de insamlade uppgifterna. Avsnitt~\ref{sec:plannedDataAnalysis}
beskriver den metod som används för dataanalysen. Slutligen, Avsnitt~\ref{sec:evaluationFramework}
beskriver ramverket som valts för att utvärdera xxx.\\
Ofta kan man koppla ett antal följdfrågor till undersökningsfrågan och problemlösningen t ex\\
(1) Vilken process skall användas för konstruktion av lösningen och vilken process skall kopplas till denna för att svara på undersökningsfrågan?\\
(2) Hur och vilket resultat (storheter) skall presenteras både för att redovisa svar på undersökningsfrågan (resultatkapitlet i denna rapport) och redovisa resultat av problemlösningen (prototypen, ofta dokument som bilagor men vilka dokument och varför?).\\
(3) Vilken teori/teknik skall väljas och användas både för undersökningen (taxonomi, matematik, grafer, storheter mm)  och  problemlösning (UML, UseCases, Java mm) och varför?\\
(4) Vad behöver du som student leverera för att uppnå hög kvaliet (minimikrav) eller mycket hög kvalitet på examensarbetet?\\
(5) Frågorna kopplar till de följande underkapitlen.\\
(6) Resonemanget bygger på att studenter på hing-programmet ofta skall konstruera något åt problemägaren och att man till detta måste koppla en intressant ingenjörsfråga. Det finns hela tiden en dualism mellan dessa aspekter i exjobbet.
}

\section{Research Process}
\label{sec:researchProcess}

\sweExpl{Undersökningsrocess och utvecklingsprocess}

Figure~\ref{fig:researchprocess} shows the steps conducted to carry out this research. 

\sweExpl{Figur~\ref{fig:researchprocess} visar de steg som utförs för att genomföra\\
Beskriv, gärna med ett aktivitetsdiagram (UML?), din undersökningsprocess och utvecklingsprocess.  Du måste koppla ihop det akademiska intresset (undersökningsprocess) med ursprungsproblemet (utvecklingsprocess)
denna forskning.\\
Aktivitetsdiagram från t ex UML-standard}


 
\begin{figure}[!ht]
  \begin{center}
    \includegraphics[width=0.5\textwidth]{figures/researchprocess.png}
  \end{center}
  \caption{Research Process}
  \label{fig:researchprocess}
\end{figure}

\generalExpl{Example of using customized item labels.}
Some steps in the process:
\begin{enumerate}[leftmargin=*, label=\textbf{Step \arabic*}, ref=Step \arabic*] %labelindent=1em for indent
    \itemsep0em
    \item \label{x:s1} plan experiment,
    \item \label{x:s2} conduct experiment,
    \item \label{x:s3} analyze data from the experiment, and
    \item \label{x:s4} discuss the results of the analysis.
\end{enumerate}

\sweExpl{Forskningsprocessen}


\section{Research Paradigm}
\label{sec:researchParadigm}
\sweExpl{Undersökningsparadigm\\
Exempelvis\\
Positivistisk (vad/hur fungerar det?) kvalitativ fallstudie med en deduktivt (förbestämd) vald ansats och ett induktivt(efterhand uppstår dataområden och data) insamlade av data och erfarenheter.}


\section{Data Collection}
\label{sec:dataCollection}
\sweExpl{Datainsamling\\
(Detta bör också visa att du är medveten om de sociala och etiska frågor som
kan vara relevanta för dina data insamlingsmetod.)}
\generalExpl{This should also show that you are aware of the social and ethical concerns that might be relevant to your data collection method.}



\subsection{Sampling}
\sweExpl{Stickprovsundersökning}

\subsection{Sample Size}
\sweExpl{Provstorleken}

\subsection{Target Population}
\sweExpl{Målgruppen}

\section[Experimental design/Planned Measurements]{Experimental design and\\Planned Measurements}
\label{sec:experimentalDesign}
\sweExpl{Experimentdesign/Mätuppställning}

\subsection{Test environment/test bed/model}
\engExpl{Describe everything that someone else would need to reproduce your test environment/test bed/model/… .}
\sweExpl{Testmiljö/testbädd/modell\\
Beskriv allt att någon annan skulle behöva återskapa din testmiljö / testbädd / modell / …}

\subsection{Hardware/Software to be used}
\sweExpl{Hårdvara / programvara som ska användas}


\section{Assessing reliability and validity of the data collected}
\label{sec:assessingReliability}
\sweExpl{Bedömning av validitet och reliabilitet hos använda metoder och insamlade data }


\subsection{Validity of method}
\label{sec:validtyOfMethod}
\sweExpl{Giltigheten av metoder\\
  Har dina metoder gett dig de rätta svaren och lösningarna? Var metoderna korrekta?}

\engExpl{How will you know if your results are valid?}
\engExpl{Remember that validity is about the \textit{accuracy} of a measurement while reliability is about the \textit{consistency} of the measurement values under the same conditions (\ie repeatability).}

\subsection{Reliability of method}
\label{sec:reliabilityOfMethod}
\sweExpl{Tillförlitlighet av för metoder\\
Hur bra är dina metoder, finns det bättre metoder? Hur kan du förbättra dem?}
\engExpl{How will you know if your results are reliable?}

\subsection{Data validity}
\label{sec:dataValidity}
\sweExpl{Giltigheten av uppgifter\\
Hur vet du om dina resultat är giltiga? Är ditt resultat rättvisande?}

\subsection{Reliability of data}
\label{sec:reliabilityOfData}
\sweExpl{Tillförlitlighet av data\\
Hur vet du om dina resultat är tillförlitliga? Hur bra är dina resultat?}


\section{Planned Data Analysis}
\label{sec:plannedDataAnalysis}
\sweExpl{Metod för analys av data}


\subsection{Data Analysis Technique}
\label{sec:dataAnalysisTechnique}
\sweExpl{Dataanalysteknik}

\subsection{Software Tools}
\label{sec:softwareTools}
\sweExpl{Mjukvaruverktyg}


\section{Evaluation framework}
\label{sec:evaluationFramework}
\sweExpl{Utvärdering och ramverk\\
Metod för utvärdering, jämförelse mm. Kopplar till kapitel~\ref{ch:resultsAndAnalysis}.}

\section{System documentation}
\label{sec:systemDocumentation}
\sweExpl{Systemdokumentation\\
Med vilka dokument och hur skall en konstruerad prototyp dokumenteras? Detta blir ofta bilagor till rapporten och det som problemägaren till det ursprungliga problemet (industrin) ofta vill ha.\\
Bland dessa bilagor återfinns ofta, och enligt någon angiven standard, kravdokument, arkitekturdokument, designdokumnet, implementationsdokument, driftsdokument, testprotokoll mm.}
\generalExpl{If this is going to be a complete document consider putting it in as an appendix, then just put the highlights here.}


\cleardoublepage
\chapter{What you did}\engExpl{Choose your own chapter title to describe this}
\label{ch:whatYouDid}
\sweExpl{[Vad gjorde du? Hur gick det till? – Välj lämplig rubrik (“Genomförande”, “Konstruktion”, ”Utveckling”  eller annat]}


\engExpl{What have you done? How did you do it? What design decisions did you make? How did what you did help you to meet your goals?}
\sweExpl{Vad du har gjort? Hur gjorde du det? Vilka designval gjorde du?\\
Hur kom det du hjälpte dig att uppnå dina mål?}

% the following sets the TOC entry to break after the & - note you have to include the first letter of the following word as it get swolled by the \texorpdfstring{}{} processing
\section[Hardware/Software design …/Model/Simulation model \&\texorpdfstring{\\}{ p} parameters/…]{Hardware/Software design …/Model/Simulation model \& parameters/…}
\sweExpl{Hårdvara / Mjukvarudesign ... / modell / Simuleringsmodell och parametrar / …}

Figure~\ref{fig:homepageicon} shows a simple icon for a home page. The time
to access this page when served will be quantified in a series of
experiments. The configurations that have been tested in the test bed are
listed in Table~\ref{tab:configstested}. In \SI{7.0}{\percent} of cases, there was an error indicating xxxxx.

\sweExpl{Figur~\ref{fig:homepageicon}  visar en enkel ikon för en hemsida. Tiden för att få tillgång till den här sidan när den laddas kommer att kvantifieras i en serie experiment. De konfigurationer som har testats i provbänk listas ini tabell~\ref{tab:configstested}.\\
Vad du har gjort? Hur gjorde du det? Vilka designval gjorde du?}
 
\begin{figure}[!ht]
  \begin{center}
    \includegraphics[width=0.25\textwidth]{figures/Homepage-icon.png}
  \end{center}
  \caption{Homepage icon}
  \label{fig:homepageicon}
\end{figure}

\begin{table}[!ht]
  \begin{center}
    \caption{Configurations tested}
    \label{tab:configstested}
    \resizebox{\columnwidth}{!}{%
    \begin{tabular}{l|c} % <-- Alignments: 1st column left, 2nd middle and 3rd right, with vertical lines in between
      \textbf{Configuration} & \textbf{Description} \\
      \hline
      1 & Simple test with one server\\
      2 & Simple test with one server\\
    \end{tabular}
    }
  \end{center}
\end{table}
\sweExpl{Testade konfigurationer}

\section{Implementation …/Modeling/Simulation/…}
\label{sec:implementationDetails}
\sweExpl{Implementering … / modellering / simulering / …}

Two commonly used simulators are:
\begin{description}[labelwidth =\widthof{\textbf{ns-2 or ns-3 simulator}}, leftmargin = !]
    \item[\textbf{Mininet}] This simulator uses traffic control (\texttt{tc}) to simulate network devices connected by links with specific bandwidth, packet loss rates, qdisc methods, etc.
    
    
    \item[\textbf{ns-2 or ns-3 simulator}] These simulators are very useful for simulating wireless communication links between moving devices. You can specify the mobility patterns of the nodes.
\end{description}

\subsection{Some examples of coding}
\engExpl{This section is simply to show some example of how you can include code in your thesis - this is not a section you would have in your thesis.}
\sweExpl{Det här avsnittet är helt enkelt för att visa ett exempel på hur du kan inkludera kod i ditt examensarbete - det här är inte ett avsnitt du skulle ha i ditt examensarbete.}

Listing~\ref{lst:helloWorldInC} shows an example of a simple program written
in C code.

\begin{lstlisting}[language={C}, caption={Hello world in C code}, label=lst:helloWorldInC]
int main() {
printf("hello, world");
return 0;
}
\end{lstlisting}


In contrast, Listing~\ref{lst:programmes} is an example of code in Python to
get a list of all of the programs at KTH.

\lstset{extendedchars=true}  %% This allows characters codes in the range 128-255
\begin{lstlisting}[language={Python}, caption={Using a python program to
    access the KTH API to get all of the programs at KTH}, label=lst:programmes]
KOPPSbaseUrl = 'https://www.kth.se'

def v1_get_programmes():
    global Verbose_Flag
    #
    # Use the KOPPS API to get the data
    # note that this returns XML
    url = "{0}/api/kopps/v1/programme".format(KOPPSbaseUrl)
    if Verbose_Flag:
        print("url: " + url)
    #
    r = requests.get(url)
    if Verbose_Flag:
        print("result of getting v1 programme: {}".format(r.text))
    #
    if r.status_code == requests.codes.ok:
        return r.text           # simply return the XML
    #
    return None
\end{lstlisting}
\FloatBarrier

\subsection{Some examples of figures in tikz}
\engExpl{This section is simply to show some example of how you can draw your own figures for in your thesis - this is not a section you would have in your thesis.}
\sweExpl{Det här avsnittet är helt enkelt för att visa ett exempel på hur du kan rita dina egna figurer i ditt examensarbete – det här är inte ett avsnitt du skulle ha i ditt examensarbete.}

These figures are just some examples to show that you can draw your own figures for in your thesis. This has two advantages: \first you do not have to worry about copyrights -- as these are your own figures and \Second the text is now readable and not simply a picture of text -- so screen readers can read the figure's contents to someone who is listening to the contents of your thesis.

\subsubsection{Azure's Form Recognizer}
\Cref{fig:processAnInvoice} shows the processing of key-value extraction from a PDF document using Azure's Form Recognizer. 

\tikzset{
    processBox/.style={rectangle, rounded corners, minimum width=3cm, minimum height=1cm,text centered, font=\sffamily, draw=black, fill=red!20},
    largeBox/.style={rectangle, rounded corners, minimum width=3cm, minimum height=4cm,text centered, draw=black}
}
\begin{figure}[!ht]
\resizebox{1.1\textwidth}{!}{%
\begin{tikzpicture}
[align=left,node distance=2cm]

\node (document) [tape,tape bend top=none,draw,font=\sffamily] {PDF\\Document};
\node (GDM) [processBox,  right=0.5cm of document] {OCR};
\node (OCRoutput) [largeBox, right=1cm of GDM] {OCR output};

\node (kvp) [tape,tape bend top=none,draw,font=\sffamily, below=0.25cm of OCRoutput.north] {key-value\\pairs};
\node (entities) [tape,tape bend top=none,draw,font=\sffamily, above=0.35cm of OCRoutput.south] {Entities};
\node (Manual) [processBox, right=1cm of kvp] {Analyze the extracted\\key-value pairs};
\draw [-latex](document) --  (GDM);
\draw [-latex](kvp) --  (Manual);
\path[ draw
     , -latex'] let \p1=(GDM.east), \p2=(kvp.west) in (GDM.east) -- +(0.25*\x2-0.25*\x1, \y1) -- +(0.5*\x2-0.5*\x1, \y2) -- (kvp.west);
\path[ draw
     , -latex'] let \p1=(GDM.east), \p2=(kvp.west), \p3=(entities.west) in (GDM.east) --  +(0.25*\x2-0.25*\x1, \y1) -- +(0.5*\x3-0.5*\x1, \y3) -- (entities.west);
\end{tikzpicture}
}
\caption{The processing of key-value extraction from a PDF document using Azure's Form Recognizer}
  \label{fig:processAnInvoice}
\end{figure}
\FloatBarrier
\subsubsection{Hyper-V with Containers}
 \Cref{fig:hyperVcontainers} shows how Hyper-V deals with containers.
 
 \tikzset{
    container/.style={rectangle, rounded corners, minimum width=2cm, minimum height=1cm,text centered, draw=black, fill=blue!20},
    containerization/.style={rectangle, rounded corners, minimum width=13.25cm, minimum height=1cm,text centered, draw=black, fill=blue!20},
    hypervisor/.style={rectangle, rounded corners, minimum width=13.25cm, minimum height=1cm,text centered, draw=black, fill=red!20},
    os/.style={rectangle, rounded corners, minimum width=13.25cm, minimum height=1cm,text centered, draw=black, fill=orange!20},
    guestos/.style={rectangle, rounded corners, minimum width=2cm, minimum height=1cm,text centered, draw=black, fill=orange!40},
    infrastructure/.style={rectangle, rounded corners, minimum width=13.25cm, minimum height=1cm,text centered, draw=black, fill=green!20},
    hos/.style={rectangle, rounded corners, minimum width=6cm, minimum height=1cm,text centered, draw=black, fill=orange!20},
    kernel/.style={rectangle, rounded corners, minimum width=6cm, minimum height=1cm,text centered, draw=black, fill=purple!20},
    services/.style={rectangle, rounded corners, minimum width=3cm, minimum height=1cm,text centered, draw=black, fill=pink!20]}
}

\begin{figure}[ht!]
    \centering
\resizebox{1\textwidth}{!}{%
\begin{tikzpicture}
[align=center,node distance=2cm]

\node (Infrastructure) [infrastructure, text width=13cm, text centered] {Infrastructure};
\node (OS1) [hos, anchor=north west, align=left, above=1.5cm of Infrastructure.north west, anchor=north west, text width=6cm, text centered] {Host OS};

\node (OS2) [hos, anchor= west, align=left, right=0.5cm of OS1.east, text width=6cm, anchor= west, text centered] {Host OS};

\node (Kernel1) [kernel, anchor=north west, align=left, above=1.5cm of OS1.north east, anchor=north east, text width=3cm, text centered] {Kernel};

\node (Kernel2) [kernel, anchor=north west, align=left, above=1.5cm of OS2.north east, anchor=north east, text width=3cm, text centered] {Kernel};

\node (ServiceA) [container, anchor=east, above=1 cm of Kernel1.east, anchor=east] {Services};
\node (AppA) [container,  left=0.25cm of ServiceA] {App 1};

\node (ServiceB) [container, anchor=east, above=1 cm of Kernel2.east, anchor=east] {Services};
\node (AppB) [container,  left=0.25cm of ServiceB] {App 2};
%\node (AppC) [container,  right=0.25cm of AppB] {App 3};

\draw[black,thick,dashed] ($(OS2.north west)+(-0.3,3.75)$)  rectangle ($(OS2.south east)+(0.5,-0.3)$);
\node[text width=5cm, text=red, above=0.1cm of ServiceB] 
    {\textbf{Container}};

\draw[red,thick,dotted] ($(Kernel2.north west)+(-0.3,1.6)$)  rectangle ($(Kernel2.south east)+(0.3,-0.3)$);
\node[text width=5cm, text=black, above=0.8cm of ServiceB] 
    {\textbf{VM}};
\end{tikzpicture}
}
    \caption{Hyper-V with containers}
    \label{fig:hyperVcontainers}
\end{figure}
\FloatBarrier
\subsubsection{\glsfmtshort{VM} versus Containers}
\Cref{fg:vmsVersusContainers} shows a comparison of virtual machines (VMs) versus containers.

\begin{figure*}[ht!]
    \centering
    \begin{subfigure}[t]{0.5\textwidth}
        \centering
\resizebox{1\textwidth}{!}{%
\begin{tikzpicture}
[align=left,node distance=2cm]

\node (AppA) [container,align=left] {App 1};
\node (AppB) [container,  right=0.25cm of AppA] {App 2};
\node (AppC) [container,  right=0.25cm of AppB] {App 3};

\node (GosA) [guestos,align=left,  below=0.25cm of AppA.south west,anchor=north west] {Guest OS};
\node (GosB) [guestos,  right=0.25cm of GosA] {Guest OS};
\node (GosC) [guestos,  right=0.25cm of GosB] {Guest OS};

\draw [decoration={brace,amplitude=0.5em},decorate, ultra thick,gray, transform canvas={xshift = 0.5cm}]
       (AppC.north -| AppC.east) -- (GosC.south -| AppC.east);
\node[text width=5cm,  right=1cm of GosC.north east] 
    {\textbf{VMs}};

\node (Hypervisor) [hypervisor, anchor=north west, align=left, below=0.25cm of GosA.south west, anchor=north west, text width=13cm, text centered] {Hypervisor};

\node (OS) [os, anchor=north west, align=left, below=0.25cm of Hypervisor.south west, anchor=north west, text width=13cm, text centered] {Host OS};

\node (Infrastructure) [infrastructure, anchor=north west, align=left, below=0.25cm of OS.south west, anchor=north west, text width=13cm, text centered] {Infrastructure};


\end{tikzpicture}
}
        \caption{VM}
    \end{subfigure}%
    ~ 
    \begin{subfigure}[t]{0.5\textwidth}
        \centering
        \resizebox{1\textwidth}{!}{%
\begin{tikzpicture}
[align=left,node distance=2cm]

\node (AppA) [container,align=left] {App 1};
\node (AppB) [container,  right=0.25cm of AppA] {App 2};
\node (AppC) [container,  right=0.25cm of AppB] {App 3};
\node[text width=5cm,  right=0.25cm of AppC] 
    {\textbf{Apps running in Containers}};


\node (Containerization) [containerization, anchor=north west, align=left, below=0.25cm of AppA.south west, anchor=north west, text width=13cm, text centered] {Docker Engine};

\node (OS) [os, anchor=north west, align=left, below=0.25cm of Containerization.south west, anchor=north west, text width=13cm, text centered] {Host OS};

\node (Infrastructure) [infrastructure, anchor=north west, align=left, below=0.25cm of OS.south west, anchor=north west, text width=13cm, text centered] {Infrastructure};


\end{tikzpicture}
}
        \caption{Containers}
    \end{subfigure}
    \caption{Virtual machines (VMs) versus Containers}
    \label{fg:vmsVersusContainers}
\end{figure*}

\cleardoublepage
\chapter{Results and Analysis}
\label{ch:resultsAndAnalysis}
\sweExpl{svensk: Resultat och Analys}

\engExpl{Sometimes this is split into two chapters.\\Keep in mind: How you are going to evaluate what you have done? What are your metrics?\\Analysis of your data and proposed solution\\Does this meet the goals which you had when you started?}

In this chapter, we present the results and discuss them.

\sweExpl{I detta kapitel presenterar vi resultaten och diskutera dem.\\Ibland delas detta upp i två kapitel.\\Hur du ska utvärdera vad du har gjort? Vad är din statistik?\\Analys av data och föreslagen lösning\\Innebär detta att uppfyllelse av de mål som du hade när du började?}

\section{Major results}
\sweExpl{Huvudsakliga resultat}

Some statistics of the delay measurements are shown in Table~\ref{tab:delayMeasurements}.
The delay has been computed from the time the GET request is received until the response is sent.

\sweExpl{Lite statistik av fördröjningsmätningarna visas i Tabell~\ref{tab:delayMeasurements}. Förseningen har beräknats från den tidpunkt då begäran GET tas emot fram till svaret skickas.}

\begin{table}[!ht]
  \begin{center}
    \caption{Delay measurement statistics}
    \label{tab:delayMeasurements}
    \begin{tabular}{l|S[table-format=4.2]|S[table-format=3.2]} % <-- Alignments: 1st column left, 2nd middle and 3rd right, with vertical lines in between
      \textbf{Configuration} & \textbf{Average delay (ns)} & \textbf{Median delay (ns)}\\
      \hline
      1 & 467.35 & 450.10\\
      2 & 1687.5 & 901.23\\
    \end{tabular}
  \end{center}
\end{table}

Table \ref{tab:ping_results} shows the measurement of round trip times from four hosts to and from a server.
\begin{table}[ht!]
\caption[RTT for 4 hosts]{Result for the ping measurements of RTT for 4 hosts} 
\label{tab:ping_results}
\vspace{1em}
\centering
\begin{tabular}{l *{4}{S[table-format=2.3]}}
{} & \multicolumn{4}{c}{host to server RTT in ms} \\
\cmidrule{2-5}
Host & \multicolumn{1}{c}{min}  & \multicolumn{1}{c}{avg} & \multicolumn{1}{c}{max} & \multicolumn{1}{c}{mdev} \\
\midrule
h1 & 5.625 & 5.625 & 5.625 & 0.0 \\
h2 & 2.909 & 2.909 & 1.909 & 0.0 \\
h3 & 5.007 & 5.007 & 5.007 & 0.0 \\
h4 & 2.308 & 2.308 & 2.308 & 0.0 \\
\midrule
\end{tabular}
\end{table}
\FloatBarrier

\sweExpl{Fördröj mätstatistik}
\sweExpl{Konfiguration | Genomsnittlig fördröjning (ns) | Median fördröjning (ns)}

Figure \ref{fig:processing_vs_payload_length} shows an example of the performance as measured in the experiments.

\begin{figure}[!ht]
% GNUPLOT: LaTeX picture
\setlength{\unitlength}{0.240900pt}
\ifx\plotpoint\undefined\newsavebox{\plotpoint}\fi
\begin{picture}(1500,900)(0,0)
\sbox{\plotpoint}{\rule[-0.200pt]{0.400pt}{0.400pt}}%
\put(171.0,131.0){\rule[-0.200pt]{4.818pt}{0.400pt}}
\put(151,131){\makebox(0,0)[r]{ 1.5}}
\put(1419.0,131.0){\rule[-0.200pt]{4.818pt}{0.400pt}}
\put(171.0,212.0){\rule[-0.200pt]{4.818pt}{0.400pt}}
\put(151,212){\makebox(0,0)[r]{ 2}}
\put(1419.0,212.0){\rule[-0.200pt]{4.818pt}{0.400pt}}
\put(171.0,292.0){\rule[-0.200pt]{4.818pt}{0.400pt}}
\put(151,292){\makebox(0,0)[r]{ 2.5}}
\put(1419.0,292.0){\rule[-0.200pt]{4.818pt}{0.400pt}}
\put(171.0,373.0){\rule[-0.200pt]{4.818pt}{0.400pt}}
\put(151,373){\makebox(0,0)[r]{ 3}}
\put(1419.0,373.0){\rule[-0.200pt]{4.818pt}{0.400pt}}
\put(171.0,454.0){\rule[-0.200pt]{4.818pt}{0.400pt}}
\put(151,454){\makebox(0,0)[r]{ 3.5}}
\put(1419.0,454.0){\rule[-0.200pt]{4.818pt}{0.400pt}}
\put(171.0,534.0){\rule[-0.200pt]{4.818pt}{0.400pt}}
\put(151,534){\makebox(0,0)[r]{ 4}}
\put(1419.0,534.0){\rule[-0.200pt]{4.818pt}{0.400pt}}
\put(171.0,615.0){\rule[-0.200pt]{4.818pt}{0.400pt}}
\put(151,615){\makebox(0,0)[r]{ 4.5}}
\put(1419.0,615.0){\rule[-0.200pt]{4.818pt}{0.400pt}}
\put(171.0,695.0){\rule[-0.200pt]{4.818pt}{0.400pt}}
\put(151,695){\makebox(0,0)[r]{ 5}}
\put(1419.0,695.0){\rule[-0.200pt]{4.818pt}{0.400pt}}
\put(171.0,776.0){\rule[-0.200pt]{4.818pt}{0.400pt}}
\put(151,776){\makebox(0,0)[r]{ 5.5}}
\put(1419.0,776.0){\rule[-0.200pt]{4.818pt}{0.400pt}}
\put(171.0,131.0){\rule[-0.200pt]{0.400pt}{4.818pt}}
\put(171,90){\makebox(0,0){ 0}}
\put(171.0,756.0){\rule[-0.200pt]{0.400pt}{4.818pt}}
\put(298.0,131.0){\rule[-0.200pt]{0.400pt}{4.818pt}}
\put(298,90){\makebox(0,0){ 10}}
\put(298.0,756.0){\rule[-0.200pt]{0.400pt}{4.818pt}}
\put(425.0,131.0){\rule[-0.200pt]{0.400pt}{4.818pt}}
\put(425,90){\makebox(0,0){ 20}}
\put(425.0,756.0){\rule[-0.200pt]{0.400pt}{4.818pt}}
\put(551.0,131.0){\rule[-0.200pt]{0.400pt}{4.818pt}}
\put(551,90){\makebox(0,0){ 30}}
\put(551.0,756.0){\rule[-0.200pt]{0.400pt}{4.818pt}}
\put(678.0,131.0){\rule[-0.200pt]{0.400pt}{4.818pt}}
\put(678,90){\makebox(0,0){ 40}}
\put(678.0,756.0){\rule[-0.200pt]{0.400pt}{4.818pt}}
\put(805.0,131.0){\rule[-0.200pt]{0.400pt}{4.818pt}}
\put(805,90){\makebox(0,0){ 50}}
\put(805.0,756.0){\rule[-0.200pt]{0.400pt}{4.818pt}}
\put(932.0,131.0){\rule[-0.200pt]{0.400pt}{4.818pt}}
\put(932,90){\makebox(0,0){ 60}}
\put(932.0,756.0){\rule[-0.200pt]{0.400pt}{4.818pt}}
\put(1059.0,131.0){\rule[-0.200pt]{0.400pt}{4.818pt}}
\put(1059,90){\makebox(0,0){ 70}}
\put(1059.0,756.0){\rule[-0.200pt]{0.400pt}{4.818pt}}
\put(1185.0,131.0){\rule[-0.200pt]{0.400pt}{4.818pt}}
\put(1185,90){\makebox(0,0){ 80}}
\put(1185.0,756.0){\rule[-0.200pt]{0.400pt}{4.818pt}}
\put(1312.0,131.0){\rule[-0.200pt]{0.400pt}{4.818pt}}
\put(1312,90){\makebox(0,0){ 90}}
\put(1312.0,756.0){\rule[-0.200pt]{0.400pt}{4.818pt}}
\put(1439.0,131.0){\rule[-0.200pt]{0.400pt}{4.818pt}}
\put(1439,90){\makebox(0,0){ 100}}
\put(1439.0,756.0){\rule[-0.200pt]{0.400pt}{4.818pt}}
\put(171.0,131.0){\rule[-0.200pt]{0.400pt}{155.380pt}}
\put(171.0,131.0){\rule[-0.200pt]{305.461pt}{0.400pt}}
\put(1439.0,131.0){\rule[-0.200pt]{0.400pt}{155.380pt}}
\put(171.0,776.0){\rule[-0.200pt]{305.461pt}{0.400pt}}
\put(30,453){\rotatebox{-270}{\makebox(0,0){Processing time (ms)}}}
\put(805,29){\makebox(0,0){Payload size (bytes)}}
\put(868.0,131.0){\rule[-0.200pt]{0.400pt}{84.074pt}}
\put(995.0,131.0){\rule[-0.200pt]{0.400pt}{98.287pt}}
\put(1173.0,131.0){\rule[-0.200pt]{0.400pt}{118.041pt}}
\put(1325.0,131.0){\rule[-0.200pt]{0.400pt}{134.904pt}}
\put(1350.0,131.0){\rule[-0.200pt]{0.400pt}{137.795pt}}
\put(1439.0,131.0){\rule[-0.200pt]{0.400pt}{155.380pt}}
\end{picture}
\caption[A GNUplot figure]{Processing time vs. payload length}\vspace{0.5cm}
\label{fig:processing_vs_payload_length}
\end{figure}
\FloatBarrier		

Given these measurements, we can calculate our processing bit rate as the inverse of the time it takes to process an additional byte divided by 8 bits per byte:

\[
	\text{bit rate} = \frac{1}{\frac{\text{time}_{\text{byte}}}{8}} = 20.03 \quad kb/s
\] 

\Cref{tab:majorMarkupLMDetailedResult} shows another table in which some values have been set in bold (using \textbackslash B) to emphasize them. Note how the \texttt{S} formatting has been modified so that it considers the weight of the characters and this is able to decimal align even these hold-faced numbers with the numbers in the column above them.

\begin{table}[!ht]
    \centering
    \caption{Median values of sandwich attributes}
    \label{tab:majorMarkupLMDetailedResult}
    \begin{tabular}{l *{2}{S[detect-weight,mode=text,table-format=3.2]}}
        & \multicolumn{2}{c}{\textbf{sites}}\\
        \cmidrule{2-3}
        \textbf{Attribute} & \textbf{A} & \textbf{B} \\
        \midrule
        price (in SEK) & 36.5 & 71.3 \\
        protean (g) & 97.2 & 100.0 \\
        salt (mg) & 9.7 & 9.3 \\
        \hline
        \textbf{Average customer rating in \%} & \B 82.2 & \B 89.9 \\
        \midrule
    \end{tabular}
\end{table}
\FloatBarrier


\Needspace*{4\baselineskip}
\Cref{fig:stackedrust} shows a stacked bar chart using pgfplots. It illustrates how easy it is to take a set of data and make a stacked bar plot. One of the features is the shifted values -- this is very useful when the bar itself is too small to put the value into.

\pgfplotstableread{
Label Numbers  Refs  Struct/Enum  Heap  Arrays
cratesio 70.04 19.83 8.31 1.3 0.52
librs 49.26 30.49 10.80 7.92 1.53
rustc 55.01 24.80 11.54 6.16 2.49
}\testdata


\pgfkeys{
    /pgf/number format/.cd,
    fixed,
    fixed zerofill,
    precision=2
}
\begin{figure}[ht!]
    \centering
    \scalebox{0.9}{
    \begin{tikzpicture}
    \begin{axis}[
        ybar stacked,
        %reverse legend,
        reverse legend=false,
        %https://tex.stackexchange.com/questions/88892/pgfplots-bar-plot-spacing-inbetween-bars
        enlarge x limits=0.4,
	    bar width=45pt,
        /pgfplots/nodes near coords*/.append style={
        every node near coord/.style={
            color=black,
            font=\small,
            name=X,
%            shift={    
%                (50pt,25pt)
%                },
            xshift={50pt},
                yshift ={
                ifthenelse((\plotnum == 4), 30pt,20pt)},
            },
            scatter/@post marker code/.append code={
                \node(Y){};
                \draw(X)--(Y.center);
            }
        },
	    nodes near coords,
        bar shift=5pt,
        ymin=0,
        ymax=115,
        xtick=data,
        width=1\textwidth,
        legend style={draw=none},
        legend image post style={scale=2.0},
        legend style={
            at={(0.5,-0.2)},
            anchor=north,
            legend columns=-2,
            font=\large,
            %mark size=20pt,
        },
        ylabel=Percentage points (\%),
        xticklabels from table={\testdata}{Label},
        xticklabel style={rotate=30},
    ]
    \addplot  table [y=Numbers, meta=Label, x expr=\coordindex] {\testdata};
    \addlegendentry{Numbers}
    \addplot table [y=Refs, meta=Label, x expr=\coordindex] {\testdata};
    \addlegendentry{Refs}
    \addplot  table [y=Struct/Enum, meta=Label, x expr=\coordindex] {\testdata};
    \addlegendentry{Struct/Enum}
    \addplot  table [y=Heap, meta=Label, x expr=\coordindex] {\testdata};
    \addlegendentry{Heap}
    \addplot  table [y=Arrays, meta=Label, x expr=\coordindex] {\testdata};
    \addlegendentry{Arrays}
    \end{axis}
    \end{tikzpicture}}
\caption{Rust types distribution for the compiler, crates.io, and lib.rs.
(percentage) - appears here with the permission of the author - see the thesis at \url{https://urn.kb.se/resolve?urn=urn\%3Anbn\%3Ase\%3Akth\%3Adiva-332124}}
\label{fig:stackedrust}
\end{figure}
\FloatBarrier



\section{Reliability Analysis}
\sweExpl{Analys av tillförlitlighet\\
Tillförlitlighet i metod och data}

\section{Validity Analysis}
\sweExpl{Analys av validitet\\
Validitet i metod och data}

\cleardoublepage
\chapter{Discussion}
\label{ch:discussion}
\sweExpl{Diskussion\\
Förbättringsförslag?}
\generalExpl{This can be a separate chapter or a section in the previous chapter.}

\cleardoublepage
\chapter{Conclusions and Future work}
\label{ch:conclusionsAndFutureWork}
\sweExpl{Slutsats och framtida arbete}

\generalExpl{Add text to introduce the subsections of this chapter.}

\section{Conclusions}
\label{sec:conclusions}
\sweExpl{Slutsatser}
\engExpl{Describe the conclusions (reflect on the whole introduction given in Chapter 1).}


  
\engExpl{Discuss the positive effects and the drawbacks.\\
Describe the evaluation of the results of the degree project.\\
Did you meet your goals?\\
What insights have you gained?\\
What suggestions can you give to others working in this area?\\
If you had it to do again, what would you have done differently?}

\sweExpl{Uppfyllde du dina mål?\\
Vilka insikter har du fått?\\
Vilka förslag kan du ge till andra som arbetar inom detta område?
Om du skulle göra detta igen, vad skulle du ha gjort annorlunda?}

\section{Limitations}
\label{sec:limitations}
\sweExpl{Begränsande faktorer\\Vad gjorde du som begränsade dina ansträngningar? Vilka är begränsningarna i dina resultat?}
\engExpl{What did you find that limited your efforts? What are the limitations of your results?}


\section{Future work}
\label{sec:futureWork}
\sweExpl{Vad du har kvar ogjort?\\Vad är nästa självklara saker som ska göras?\\Vad tips kan du ge till nästa person som kommer att följa upp på ditt arbete?}
\engExpl{Describe valid future work that you or someone else could or should do.\\
Consider: What you have left undone? What are the next obvious things to be done? What hints can you give to the next person who is going to follow up on your work?}



Due to the breadth of the problem, only some of the initial goals have been
met. In these section we will focus on some of the remaining issues that
should be addressed in future work. ...

\subsection{What has been left undone?}
\label{what-has-been-left-undone}

The prototype does not address the third requirment, \ie a yearly unavailability of less than 3 minutes; this remains an open problem. ...

\subsubsection{Cost analysis}
\generalExpl{Example of a missing component}
The current prototype works, but the performance from a cost perspective makes this an impractical solution. Future work must reduce the cost of this solution; to do so, a cost analysis needs to first be done. ...

\subsubsection{Security}
\generalExpl{Example of a missing component}
A future research effort is needed to address the security holes that results from using a self-signed certificate. Page filling text mass. Page filling text mass. ...


\subsection{Next obvious things to be done}

In particular, the author of this thesis wishes to point out xxxxxx remains as a problem to be solved. Solving this problem is the next thing that should be done. ...

\section{Reflections}
\label{sec:reflections}
\sweExpl{Reflektioner}
\sweExpl{Vilka är de relevanta ekonomiska, sociala, miljömässiga och etiska aspekter av ditt arbete?}
\engExpl{What are the relevant economic, social,
  environmental, and ethical aspects of your work?
}



One of the most important results is the reduction in the amount of
energy required to process each packet while at the same time reducing the
time required to process each packet.




\noindent\rule{\textwidth}{0.4mm}
\engExpl{In the references, let Zotero or other tool fill this in for you. I suggest an extended version of the IEEE style, to include URLs, DOIs, ISBNs, etc., to make it easier for your reader to find them. This will make life easier for your opponents and examiner. \\IEEE Editorial Style Manual: \url{https://www.ieee.org/content/dam/ieee-org/ieee/web/org/conferences/style_references_manual.pdf}}
\sweExpl{Låt Zotero eller annat verktyg fylla i det här för dig. Jag föreslår en utökad version av IEEE stil - att inkludera webbadresser, DOI, ISBN osv. - för att göra det lättare för läsaren att hitta dem. Detta kommer att göra livet lättare för dina opponenter och examinator.}

\cleardoublepage
% Print the bibliography (and make it appear in the table of contents)
\renewcommand{\bibname}{References}


\ifbiblatex
    %\typeout{Biblatex current language is \currentlang}
    \printbibliography[heading=bibintoc]
\else
    \phantomsection  % make it include a hyperref - see https://tex.stackexchange.com/a/98995
    \addcontentsline{toc}{chapter}{References}
    \bibliography{references}
\fi


\end{document}
